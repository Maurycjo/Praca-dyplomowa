\chapter{Wykorzystane technologie i narzędzia}
Rozwijana aplikacja powstawnie w oparciu o architekturę klient-serwer. Podczas implementacji tego typu aplikacji wyróżnia się technologie wykorzystywane do realizacji logiki biznesowej oraz technologie do realizacji interfejsu użytkownika. Ponadto sama implementacja odbywa się w środowisku deweloperskim, zapewniającym dostęp do niezbędnych narzędzi. Zdarza się jednak i tak, że te same technologie i narzędzia pojawiają się w na każdym etapie i przy implementacji różnych części aplikacji.
W tabeli~\ref{tab:zestawienie_narzędzi} zamieszczono zestawienie głównych narzędzi i technologii wykorzystanych podczas realizacji pracy.
W kolejnych podrozdziałach podano szczegóły.


\begin{table}[htb] \small
	\centering
\caption{Zestawienie wykorzystanych narzędzi i technologii wraz z ich wersjami i licencjami}
\label{tab:zestawienie_narzędzi}
\begin{tabularx}{\linewidth}{|X|X|X|}
    \hline
    Technologia & Wersja & Licencja \\
    \hline \hline
    Java & 17 Oracle OpenJDK 17.0.8 17 &  NFTC\\
    \hline
    Maven & 3.9.5 & Apache License\\
    \hline
    Springboot & 3.1.4 & Apache 2.0\\
    \hline
    Hibernate & 6.1.7.Final & GNU Lesser General Public License \\
    \hline
		Javascript & & \\
    \hline
		SQL Server Managment Studio & Management Studio 19 Standard Edition & Microsoft Software License\\

    \hline
\end{tabularx}
\end{table}

\section{Warstwa logiki biznesowej}
\subsection{Java}
Java to termin, który w dziedzinie informatyki kojarzony jest z platformą do budowy aplikacji(ang. Java PLatform), obiektowym językiem programowania(ang. Java language) oraz z wirtualną maszyną (ang.~\emph{Java Virtual Machine}). Została wprowadzona przez firmę Sun Microsystems w roku 1995. Jest rozwijana aż do dzisiaj i doczekała się wielu wersji. 

Rozwój Javy odbywa się w ramach JCP (ang.~\emph{Java Community Process}). Każdy użytkownik może brać udział w recenzowaniu i dostarczaniu informacji zwrotnej dla publikowanych specyfikacji JSR (ang.~\emph{Java Specification Request}). Specyfikacje składają się z powiązanych ze sobą dokumentów w których skład wchodzą specyfikacja platformy, specyfikacja języka i specyfikacja maszyny wirtualnej. Na podstawie tych specyfikacji wydawcy tworzą własne implementacje Javy.

\subsubsection{Język Java}
Java jako język jest współbieżnym, opartym na klasach, obiektowym językiem programowania. Java wykorzystuje tworzenie programów źródłowych kompilowanych do kodu bajtowego. Kod bajtowy jest interpretowany przez maszynę wirtualną Javy.

\subsubsection{Maszyna Wirtualna Javy}
Maszyna wirtualna Javy jest zestawem aplikacji napisanych na tradycyjne urządzenia i systemy operacyjne. Jest środowiskiem  zdolnym do wykonywania kodu bajtowego Javy. Oferuje też automatyczne zarządzanie pamięcią przez garbage colector. Implementacja wirtualnej maszyny jest napisana w C++. Dostępna publicznie specyfikacja umożliwiła różnym producentom na tworzenie własnych wirtualnych maszyn przez co istnieje wiele różnych niezależnych wersji. To co łączy te wersje jest specyfikacja języka Java

\subsubsection{Platforma Java}
Java jako platforma oferuje programy, narzędzia które wspierają budowę i uruchomienie aplikacji. Aplikacja napisana w ramach pracy dyplomowej została stworzona w wersji Java 17. Wersja ta domyślnie została zainstalowana w środowisku programistycznym. Jest to wersja stabilna.

\section{Frameworki Java}
% TO DO: dodać tytułem wstępu parę słów, czym są frameworki

\subsection{Spring oraz Spring Boot}
% TO DO: jeśli to cytowanie, to powinno być zrobione przez \cite{} (Craig Walls: Spring w akcji. Wydanie V pdf)
Spring Framework jest platformą która ma na celu uproszczenie tworzenia oprogramowania dla platformy Java. Spring oferuje kontener któy jest określany mianem kontekstu aplikacji Springa. Odpowiedzialny jest za utworzenie komponentów aplikacji i zarządzanie nimi. Komponenty są powiązane w kontekście aplikacji i tworzą spójną całość. Operacja powiązania ze sobą komponentów bean jest oparta na wzorcu znanym jako wstrzykiwanie zależności. Zamiast zajmować się obsługą cyklu życiowego zależnych komponentów bean, w przypadku aplikacji stosującej wstrzykiwanie zależności następuje zdefiniowanie oddzielnej encji(kontenera) przeznaczonej do utworzenia i przechowywania wszystkich komponentów, które będą wstrzykiwane do potrzebujących ich komponentów bean. Najczęściej odbywa się to za pomocą argumentów konstruktora lub metod akcesora właściwości. 

Spring Boot jest rozszerzeniem Springa który eliminuje konieczność konfiguracji środowiska charakterystyczną dla samego Springa. Oferuje on także wiele ciekawych funkcjonalności takich jak analiza wewnętrznego sposobu działania aplikacji w środowisku uruchomieniowym, elastyczną specyfikacje właściwości środowiska oraz dodatkowe możliwości w zakresie obsługi testów

\subsubsection{Spring Data}
Podstawowy framework Spring jest dostarczony z prostymi możliwościami w zakresie obsługi trwałego magazynu danych, natomiast Spring Data oferuje funkcje która pozwala na zdefiniowaniu repozytorium danych aplikacji w postaci interfejsów Javy, używając konwencji nazw pod czas definiowania metod określających sposób przechowywania i pobierania danych. Spring Data pozwala na pracę z wieloma rodzajami baz danych m.in. relacyjnym użyciem API JPA(ang. java persistance api)
% TO DO: a co z innymi pakietami (SPRING WEB)


\subsection{Hibernate}
Hibernate jest frameworkiem służącym do realizacji warstwy dostępu do danych. Zapewnia on przede wszystkim translacje danych pomiędzy relacyjną bazą danych a światem obiektowym. Hibernate pozwala na automatyczne mapowanie obiektów Javy na wiersze z bazy danych oraz odczytywać rekordy z bazy i automatycznie tworzyc z nich obiekty. Wykorzystując Hibernate nie trzeba poświęcać uwagi na zapytania SQL do bazy danych bo framework robi to sam.


\section{Warstwa interfejsu użytkownika}
\subsection{Javascript}
\subsection{Arkusze styli}

\section{Środowisko deweloperskie}
\subsection{Maven}
Apache Maven jest narzędziem oferującym automatyczną budowę oprogramowania na platformę Java. Poszczególne funkcje Mavena są realizowane poprzez wtyczki. Proces budowania kończy się osiągnięciem wybranego przez budującego celu. Cele umożliwiają między innymi na kompilacje, zbudowanie paczki dystrybucyjnej oraz na uruchomienie testów automatycznych. Maven korzysta z pliku POM(ang. Project Object Model). POM to dokument XML-owy o nazwie pom.xml który opisuje projekt. W pliku tym można zdefiniować zależności które będą wykorzystane w projekcie. 

\subsection{Intelij Idea}
Intelij jest zintegrowanym środowiskiem programistycznym który oferuje wiele narzędzi pomagającym w programowaniu. W wersji ultimate która dostępna jest na licencji studenckiej występuje wsparcie do integacji i zarządzania bazą danych

\subsection{SQL Server Managament Studio}

\subsection{Postman}





\chapter{Narzędzia i Technologie}


\sectionn{Wykorzystane technologie}

\begin{tabular}{|c|c|c|}
    \hline
    Technologia & Wersja & Licencja \\
    \hline
    Java & 17 Oracle OpenJDK version 17.0.8 17 &  NFTC\\
    \hline
    Maven & Wiersz 2, Kolumna 2 & Wiersz 2, Kolumna 3 \\
    \hline
    Springboot & 3.1.4 & Apache 2.0\\
    \hline
    Hibernate & 6.1.7.Final & GNU Lesser General Public License \\
    \hline
		SQL & Microsoft SQL & \\
    \hline
\end{tabular}


\subsection{Java 17}
Java jest zorientowanym obiektowo językiem programowania. JDK implemantacje, JVM, kod bajtowy, platforma, wiele wersji, użyłem Oracle bo był domyślnie w intelij w wersji 17 która jest stabilna
Java jest niezależną platformą która działa niezależnie od systemu operacyjnego. Oferuje też bogate wsparcie dla wielu narzędzi i frameworków. Dodatkowo można z Javą zintegrować bazy danych z różnych wydań. Dzięki 

\subsection{Maven}

\subsection{Springboot}
\subsection{Hibernate}
\subsection{SQL}


\section{Wykorzystane narzędzia}

\begin{tabular}{|c|c|c|}
    \hline
    Narzędzie & Wersja & Licencja \\
    \hline
    Intelij &  &  NFTC\\
    \hline
    SQL Server Managment Studio & Wiersz 2, & \\
    \hline
    JPA Buddy & & \\
    \hline
		Draw.io & &\\
    \hline
		Postman & &\\
    \hline
\end{tabular}


\subsection{Intelij Idea}

\subsection{SSMS}

\subsection{JPA Buddy}
\subsection{Draw.io}
\subsection{Postman}





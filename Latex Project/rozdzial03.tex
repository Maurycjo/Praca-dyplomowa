\chapter{Analiza wymagań}

\section{Wymagania funkcjonalne}

Projektowany system powinien zapewnieć szereg funkcji które mogły by być wykorzystywane przez użytkowników systemu.


\begin{enumerate}[label={\textbf{FR}-\bfseries\arabic*}]
    \item \textbf{Zarządzanie użytkownikami}
    \begin{enumerate}[label={FR-\arabic{enumi}.\arabic*}]
        \item Rejestracja użytkownika
        \item Logowanie użytkownika
        \item Usuwanie pracownika
        \item Wylogowywanie użytkownika
				\item Przeglądanie zarejestrowanych pracowników
    \end{enumerate}
		
		\item \textbf{Zarządzanie sprzętem komputerowym}
    \begin{enumerate}[label={FR-\arabic{enumi}.\arabic*}]
        \item Przeglądanie sprzętu komputerowego
        \item Dodawanie nowego sprzętu komputerowego
        \item Modyfikacja sprzętu komputerowego
        \item Usuwanie sprzętu komputerowego
				\item Wyświetlanie szczegółowych informacji dotyczących sprzętu komputerowego
		\end{enumerate}
		
		\item \textbf{Zarządzanie komponentami komputera}
			\begin{enumerate}[label={FR-\arabic{enumi}.\arabic*}]		
				\item Przeglądanie komponentów komputera
				\item Dodawanie komponentu(procesora, ram, dysku) dla komputera
				\item Modyfikacja komponentu komputera
				\item Usuwanie komponentu komputera
    \end{enumerate}
		
		
		\item \textbf{Zarządzanie loterią wybranego sprzętu komputerowego}
		\begin{enumerate}[label={FR-\arabic{enumi}.\arabic*}]
        \item Zmiana gotowości sprzętu do losowania
        \item Zmiana statusu sprzętu na sprzedane lub niesprzedane
				\item Wyświetlanie listy uczestników biorących udział w losowaniu
        \item Losowanie zwycięzcy loterii
    \end{enumerate}
		
		\item \textbf{Zarządzanie uczestnictem w loterii}
			\begin{enumerate}[label={FR-\arabic{enumi}.\arabic*}]
        \item Przeglądanie historii loterii wybranego użytkownika
        \item Zapisywanie się na loterię urządzenia
				\item Wypisywanie się z loterii urządzenia
    \end{enumerate}
		
		
		
\end{enumerate}

% template %
\newcommand\addrow[2]{#1 & #2\\ \hline}

\newcommand\additemizedrow[2]{#1 &
        \begin{tabenum}
            #2
        \end{tabenum}
        \\ \hline}

% making stuff convenient %
\newcommand\name[1]{\addrow{Nazwa}{#1}}
\newcommand\actor[1]{\addrow{Aktor}{#1}}
\newcommand\udescription[1]{\addrow{Opis}{#1}}
\newcommand\precondition[1]{\addrow{Warunki wstępne}{#1}}
\newcommand\scenario[1]{\additemizedrow{Scenariusz}{#1}}
\newcommand\alternateScenario[1]{\additemizedrow{Alternatywny}{#1}}
\newcommand\extend[1]{\additemizedrow{Extend}{#1}}
\newcommand\includee[1]{\additemizedrow{Include}{#1}}

\newenvironment{usecase}{\tabularx{\textwidth}{|0{wl{3cm}}|0{X}|}\hline}{\endtabularx}
\setlength{\parindent}{0em}
\setlength{\parskip}{1em}


\subsubsection{FR-1.1 Rejestracja użytkownika}

\begin{usecase}
		\udescription{Rejestracja użytkownika ze standardową rolą, przy użyciu maila, nazwy użytkownika, imienia, nazwiska, hasła i potwierdzenia hasła}
    \actor{Pracownik}
    \precondition{Brak}
    \scenario{
        \item Podanie w formularzu maila, nazwy użytkownika, Imienia, nazwiska, hasła i potwierdzenia hasła
        \item Kliknięcie przycisku zarejestruj
				\item Przekierowanie do okna logowania
    }
		\alternateScenario{
			\item [3] Użytkownik wprowadził 2 różne hasła w polach hasło i powtórz hasło przez co rejestracja nie powiodła się
			\item [4] Użytkownik o zadanym mailu lub nazwie użytkownika istnieje przez co rejestracja nie powiodła się
		}
  
     
\end{usecase}

\subsubsection{FR-1.2 Logowanie użytkownika}

\begin{usecase}
		\udescription{Logowanie użytkownika przy pomocy email lub nazwy użytkownika oraz hasła}
    \actor{Pracownik, Administrator}
    \precondition{Użytkownik został wcześniej zarejestrowany w systemie}
    \scenario{
        \item Podanie w formularzu nazwy użytkownika lub email oraz podanie hasła
        \item Kliknięcie przycisku zaloguj
				\item Nadanie użytkownikowi odpowiednich uprawnień
				\item Przekierowanie na stronę domową
    }
    \alternateScenario{
			\item [4] Podane dane w formularzu są nieprawidłowe, przez co nie następuje logowanie i pozostaje na stronie logowania
		}
\end{usecase}

\subsubsection{FR-1.3 Usuwanie pracownika}
\begin{usecase}
		\udescription{Administrator jako zarządca systemu może usunąć urzytkownika}
    \actor{Administrator}
    \precondition{Administrator został wcześniej zarejestrowany w systemie i posiada uprawnienia administratora}
		\includee{
			\item FR-1.5 Przeglądanie zarejestrowanych pracowników
		}
	
    \scenario{
        \item Kliknięcie przycisku usuń na wierszu reprezentującym pracownika
				\item Usunięcie z bazy danych pracownika oraz jego historii uczestnictwa w loteriach
    }
\end{usecase}

\subsubsection{FR-1.4 Wylogowywanie użytkownika}
\begin{usecase}
		\udescription{Wylogowywanie z systemu}
    \actor{Administrator, Pracownik}
    \precondition{Użytkownik został wcześniej zalogowany}
    \scenario{
        \item Kliknięcie przycisku wyloguj
				\item Przejście na stronę logowania
    }
\end{usecase}

\subsubsection{FR-1.5 Przeglądanie zarejestrowanych pracowników}
\begin{usecase}
		\udescription{Możliwe jest wyświetlanie wszystkich użytkowników z rolą Pracownik}
    \actor{Administrator}
    \precondition{Administrator został wcześniej zarejestrowany w systemie i posiada uprawnienia administratora}
    \scenario{
        \item Kliknięcie przycisku zarządzaj użytkownikami
				\item Wyświetlenie widoku użytkowników
    }
\end{usecase}

\subsubsection{FR-2.1 Przeglądanie sprzętu komputerowego}
\begin{usecase}
		\udescription{Administrator widzi wszystkie sprzęty dodane w systemie, natomiast pracownik tylko te któe są gotowe do loterii}
    \actor{Administrator, Pracownik}
    \precondition{Użytkownik jest zalogowany w systemie}
		\extend{
			\item Filtrowanie jaki rodzaj sprzętu ma być wyświetlany na widoku lub wybranie wszystkich rodzajów
			\item Filtrowanie sprzętu po biurze w jakim się znajdują
		}
    \scenario{
				\item Wyświetlenie widoku sprzętów komputerowych na stronie domowej, z dostępnymi operacjami zarządzania sprzętem i loteriami tych sprzętów
    }
		\alternateScenario{
			\item [1b] Użytkownik o dostępie pracownika nie ma dostępu do operacji zarządzania sprzętem
		}
\end{usecase}


\subsubsection{FR-2.2 Dodanie nowego sprzętu komputerowego}
\begin{usecase}
		\udescription{Administrator może dodawać 3 różne rodzaje urządzeń zdefiniowanych w systemie}
    \actor{Administrator}
    \precondition{Administrator został wcześniej zarejestrowany w systemie i posiada uprawnienia administratora}
		\extend{
			\item FR-3.2 Dodawanie komponentu dla komputera
		}
    \scenario{
        \item Kliknięcie przycisku odpowiedzialnego za dodanie urządzenia
        \item Wybranie z listy rozwijanej typu urządzenia które będzie dodawane
				\item System wyświetla formularz z odpowiednim rodzajem urządzenia
				\item Wpisanie oraz wybranie odpowiednich parametrów reprezentujących urządzenie
				\item Dodanie urządzenia do systemu
    }
    \alternateScenario{
			\item [4] Administrator nie znalazł w formularzu komputera; procesora, ram, lub dysku i postanowił dodać taki klikając odpowiedni przycisk: dodaj procesor, dodaj ram lub dodaj pamięć
			\item [5] Następuje teraz wykonanie przypadku FR-3.2 Dodawanie komponentu dla komputera
			\item [6] Kontynuacja uzupełniania formularza
			\item [7] Dodanie nowego komputera
			}
\end{usecase}

\subsubsection{FR-2.3 Modyfikacja sprzętu komputerowego}
\begin{usecase}
		\udescription{Możliwa jest korekta urządzeń lub podanie brakujących parametrów urządzenia}
    \actor{Administrator}
    \precondition{Administrator został wcześniej zarejestrowany w systemie i posiada uprawnienia administratora}
		\includee{
			\item FR-2.1 Przeglądanie sprzętu komputerowego
		}

    \scenario{
        \item Kliknięcie przycisku modyfikuj na odpowiednim wierszu reprezentującym urządzenie
				\item System wykrywa odpowiedni rodzaj urządzenia i wyświetla formularz z poprzednimi danymi urządzenia
				\item Wpisanie oraz wybranie odpowiednich parametrów reprezentujących urządzenie
				\item Modyfikacja urządzenia
    }
		\alternateScenario{
			
			\item [3] Administrator nie znalazł w formularzu komputera; procesora, ram, lub dysku i postanowił dodać taki klikając odpowiedni przycisk: dodaj procesor, dodaj ram lub dodaj pamięć
			\item [4] Następuje teraz wykonanie przypadku FR-3.2 Dodawanie komponentu dla komputera
			\item [5] Kontynuacja uzupełniania formularza
			\item [6] Modyfikacja komputera
		
		}
\end{usecase}

\subsubsection{FR-2.4 Usuwanie sprzętu komputerowego}
\begin{usecase}
		\udescription{Administrator może usuwać urządzenia}
    \actor{Administrator}
    \precondition{Administrator został wcześniej zarejestrowany w systemie i posiada uprawnienia administratora}
		\includee{
		\item FR-2.1 Przeglądanie sprzętu komputerowego
		}
    \scenario{
        \item Kliknięcie przycisku usuń na odpowiednim wierszu reprezentującym urządzenie
				\item Usunięcie sprzętu z systemu
    }
\end{usecase}

\subsubsection{FR-2.5 Wyświetlanie szczegółowych informacji dotyczących sprzętu}
\begin{usecase}
		\udescription{Możliwe jest szczegółowe sprawdzenie dotyczące parametrów sprzętów w specjalnym formularzu}
    \actor{Administrator, Pracownik}
    \precondition{Użytkownik został zalogowany do systemu}
		\includee{
			\item FR-2.1 Przeglądanie sprzętu komputerowego
		}
		
    \scenario{
        \item Kliknięcie przycisku informacji o urządzeniu na odpowiednim wierszu reprezentującym urządzenie
				\item System wykrywa odpowiedni rodzaj urządzenia i wyświetla formularz urządzenia z zablokowanymi polami 
    }
 
\end{usecase}



\subsubsection{FR-3.1 Przeglądanie komponentów komputera}
\begin{usecase}
		\udescription{Do lepszego oszacowania ceny sprzętu Administrator ma widok komponentów które może posiadać komputer}
    \actor{Administrator}
    \precondition{Administrator został wcześniej zarejestrowany w systemie i posiada uprawnienia administratora}
		\extend{
			\item filtrowanie komponentów po rodzaju: procesor, ram, dysk pamięci
		}
    \scenario{
        \item Kliknięcie przycisku odpowiedzialnego za widok komponentów
				\item Wyświetlenie widoku na stronie
    }
   
\end{usecase}


\subsubsection{FR-3.2 Dodawanie nowego komponentu komputera}
\begin{usecase}
		\udescription{Możliwe jest dodawanie komponentu na który składa się cena i nazwa}
    \actor{Administrator}
    \precondition{Administrator został wcześniej zarejestrowany w systemie i posiada uprawnienia administratora}
    \scenario{
				\item Wykonanie przypadku FR-3.1 Przeglądanie komponentów(procesory, ram, dyski) komputera
        \item Kliknięcie przycisku dodaj procesor, dodaj RAM lub dodaj dysk
				\item System wyświetla formularz komponentu
				\item Wpisanie nazwy i ceny komponentu
				\item Dodanie komponentu do bazy danych
    }
		\alternateScenario{
			\item [1b] Administrator wykonuje dodanie z poziomu formularza dodawania komputera
			\item [4] Administrator podał nazwę komponentu która istnieje w bazie danych
			\item [5] Następuje modyfikacja podanej ceny istniejącego komponentu zamiast dodania nowego
		}
\end{usecase}


\subsubsection{FR-3.3 Modyfikacja komponentu komputera}
\begin{usecase}
		\udescription{Możliwe jest modyfikowanie komponentu na który składa się cena i nazwa}
    \actor{Administrator}
    \precondition{Administrator został wcześniej zarejestrowany w systemie i posiada uprawnienia administratora}
		\includee{
			\item FR-3.1 Przeglądanie komponentów(procesory, ram, dyski) komputera
		}
    \scenario{
        \item Kliknięcie przycisku modyfikuj na odpowiednim wierszu reprezentującym komponent
				\item Modyfikacja komponentu w bazie danych
    }
\end{usecase}

\subsubsection{FR-3.4 Usuwanie komponentu komputera}
\begin{usecase}
		\udescription{Możliwe jest usuwanie komponentów wchodzącym w skład komputera}
    \actor{Administrator}
    \precondition{Administrator został wcześniej zarejestrowany w systemie i posiada uprawnienia administratora}
		\includee{
			\item FR-3.1 Przeglądanie komponentów(procesory, ram, dyski) komputera
		}
    \scenario{
        \item Kliknięcie przycisku usuń na odpowiednim wierszu reprezentującym komponent
				\item Usunięcie komponentu z bazy, oraz ustawienie wszystkich odwołań w komputerach do danego komponentu na null
    }
\end{usecase}



\subsubsection{FR-4.1 Zmiana gotowości sprzętu do losowania}
\begin{usecase}
		\udescription{Administrator zarządza urządzeniem i określa jego gotowość do loterii}
    \actor{Administrator}
    \precondition{Administrator został wcześniej zarejestrowany w systemie i posiada uprawnienia administratora, Urządzenie nie zostało wcześniej wylosowane w loterii}
		\includee{
			\item FR-2.1 Przeglądanie sprzętu komputerowego
		}
    \scenario{
        \item Kliknięcie przycisku gotowości do losowania na odpowiednim wierszu reprezentującym urządzenie
				\item Zmiana gotowości losowania na przeciwny do poprzedniego
    }
   
\end{usecase}


\subsubsection{FR-4.2 Zmiana statusu sprzętu na sprzedane lub niesprzedane}
\begin{usecase}
		\udescription{Administrator kontroluje które sprzęty zostały już sprzedane i dostarczone do pracownika}
    \actor{Administrator}
    \precondition{Administrator został wcześniej zarejestrowany w systemie i posiada uprawnienia administratora, Urządzenie zostało wylosowane w loterii}
		\includee{
			\item FR-2.1 Przeglądanie sprzętu komputerowego
		}
    \scenario{
        \item Kliknięcie przycisku odpowiedzialnego za zmianę statusu sprzedane na odpowiednim wierszu reprezentującym urządzenie
				\item Aktualizacja statusu sprzedane urządzenia w systemie
    }
   
\end{usecase}



\subsubsection{FR-4.3 Wyświetlanie listy uczestników biorącym udział w losowaniu}
\begin{usecase}
		\udescription{Administrator sprawdza jacy użytkownicy biorą udział w losowaniu sprzętu}
    \actor{Administrator}
    \precondition{Administrator został wcześniej zarejestrowany w systemie i posiada uprawnienia administratora, Status urządzenia został ustawiony na gotowe do losowania}
		\includee{
			\item FR-2.1 Przeglądanie sprzętu komputerowego
		}
    \scenario{
        \item Kliknięcie przycisku odpowiedzialnego za wyświetlanie listy uczestników na odpowiednim wierszu reprezentującym urządzenie
				\item Przekierowanie do widoku użytkowników biorącym udział w losowaniu wybranego sprzętu
				\item Wyświetlanie widoku uczestników losowania, oraz ich statusu dotyczącego czy już wygrali loterię
    }
\end{usecase}


\subsubsection{FR-4.4 Losowanie zwycięzcy loterii}
\begin{usecase}
		\udescription{Administrator losuje zwycięzce spośród tych pracowników którzy zapisali się a loterię}
    \actor{Administrator}
    \precondition{Administrator został wcześniej zarejestrowany w systemie i posiada uprawnienia administratora, Status urządzenia został ustawiony na gotowe do losowania, Przynajmniej jeden pracownik jest zapisany na losowanie}
		\includee{
			\item FR-2.1 Przeglądanie sprzętu komputerowego
		}
    \scenario{
        \item Kliknięcie przycisku odpowiedzialnego za losowanie zwycięzcy na odpowiednim wierszu reprezentującym urządzenie
				\item Ustawienie daty losowania na dzisiejszą
				\item Ustawienie wylosowanego pracownika jako zwycięzce losowania 
    }
   
\end{usecase}


\subsubsection{FR-5.1 Przeglądanie historii loterii wybranego użytkownika}
\begin{usecase}
		\udescription{W systemie istnieje filtrowanie dotyczących historii loterii}
    \actor{Administrator, Pracownik}
    \precondition{Użytkownik został zalogowany w systemie}
		
		\extend{
			\item Filtrowanie loterii po statusie uczestnika: wygrana, przegrana, trwająca
			\item FR-2.5 Wyświetlanie szczegółowych informacji dotyczących sprzętu komputerowego
		}
		
    \scenario{
        \item Kliknięcie przycisku pokaż historie na odpowiednim wierszu reprezentującym użytkownika
				\item Wyświetlenie sprzętu, daty loterii oraz statusu użytkownika w odniesieniu do bieżącej loterii 
    }
		\alternateScenario{
			\item [1b] Kliknięcie przycisku pokaż historie użytkownika który jest właśnie zalogowany
		}
		
\end{usecase}


\subsubsection{FR-5.2 Zapisywanie się na loterię urządzenia}
\begin{usecase}
		\udescription{Pracownik przeglądając specyfikacje sprzętu może być zainteresowany wzięciem udziału w loterii}
    \actor{Pracownik}
    \precondition{Pracownik został wcześniej zalogowany w systemie i posiada uprawnienia pracownika, sprzęt nie został wcześniej wylosowany, pracownik wcześniej nie zapisał się na loterię dotyczącego tego sprzętu}
		\includee{
			\item FR-2.1 Przeglądanie sprzętu komputerowego
		}
    \scenario{
        \item Pracownik klika przycisk weź udział na odpowiednim wierszu reprezentującym urządzenie
				\item Do systemu zostaje dodane uczestnictwo w loterii wybranego urządzenia
    }
\end{usecase}

\subsubsection{FR-5.3 Wypisanie się z loterii}
\begin{usecase}
		\udescription{Pracownik może chcieć się wypisać z loterii}
    \actor{Pracownik}
    \precondition{Pracownik został wcześniej zalogowany w systemie i posiada uprawnienia pracownika, sprzęt nie został wcześniej wylosowany, pracownik wcześniej zapisał się na loterię dotyczącego wybranego sprzętu}
		\includee{
			\item FR-2.1 Przeglądanie sprzętu komputerowego
		}
    \scenario{
        \item Pracownik klika przycisk wypisz się na odpowiednim wierszu reprezentującym urządzenie
				\item Z systemu zostaje usunięte uczestnictwo w loterii wybranego urządzenia
    }
\end{usecase}


\section{Wymagania niefunkcjonalne}

%TODO










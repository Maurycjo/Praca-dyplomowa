\chapter{Analiza wymagań}

\section{Wymagania funkcjonalne}

Projektowany system powinien zapewnieć szereg funkcji które mogły by być wykorzystywane przez użytkowników systemu.
\begin{enumerate}[label={\textbf{FR}-\bfseries\arabic*}]
    \item \textbf{Zarządzanie użytkownikami}
    \begin{enumerate}[label={FR-\arabic{enumi}.\arabic*},noparskip]
        \item Rejestracja użytkownika
        \item Logowanie użytkownika
        \item Usuwanie pracownika
        \item Wylogowywanie użytkownika
				\item Przeglądanie zarejestrowanych pracowników
    \end{enumerate}
		
		\item \textbf{Zarządzanie sprzętem komputerowym}
    \begin{enumerate}[label={FR-\arabic{enumi}.\arabic*},noparskip]
        \item Przeglądanie sprzętu komputerowego
        \item Dodawanie nowego sprzętu komputerowego
        \item Modyfikacja sprzętu komputerowego
        \item Usuwanie sprzętu komputerowego
				\item Wyświetlanie szczegółowych informacji dotyczących sprzętu komputerowego
		\end{enumerate}
		
		\item \textbf{Zarządzanie komponentami komputera}
			\begin{enumerate}[label={FR-\arabic{enumi}.\arabic*},noparskip]		
				\item Przeglądanie komponentów komputera
				\item Dodawanie komponentu(procesora, ram, dysku) dla komputera
				\item Modyfikacja komponentu komputera
				\item Usuwanie komponentu komputera
    \end{enumerate}
		
		
		\item \textbf{Zarządzanie loterią wybranego sprzętu komputerowego}
		\begin{enumerate}[label={FR-\arabic{enumi}.\arabic*},noparskip]
        \item Zmiana gotowości sprzętu do losowania
        \item Zmiana statusu sprzętu na sprzedane lub niesprzedane
				\item Wyświetlanie listy uczestników biorących udział w losowaniu
        \item Losowanie zwycięzcy loterii
    \end{enumerate}
		
		\item \textbf{Zarządzanie uczestnictem w loterii}
			\begin{enumerate}[label={FR-\arabic{enumi}.\arabic*},noparskip]
        \item Przeglądanie historii loterii wybranego użytkownika
        \item Zapisywanie się na loterię urządzenia
				\item Wypisywanie się z loterii urządzenia
    \end{enumerate}
		
		
		
\end{enumerate}





% template %
\newcommand\addrow[2]{#1 & #2\\ \hline}

\newcommand\additemizedrow[2]{#1 &
        \begin{tabenum}
            #2
        \end{tabenum}
        \\ \hline}

% making stuff convenient %
\newcommand\name[1]{\addrow{Nazwa}{#1}}
\newcommand\actor[1]{\addrow{Aktor}{#1}}
\newcommand\udescription[1]{\addrow{Opis}{#1}}
\newcommand\precondition[1]{\addrow{Warunki wstępne}{#1}}
\newcommand\scenario[1]{\additemizedrow{Scenariusz}{#1}}
\newcommand\alternateScenario[1]{\additemizedrow{Alternatywny}{#1}}
\newcommand\extend[1]{\additemizedrow{Extend}{#1}}
\newcommand\includee[1]{\additemizedrow{Include}{#1}}

\newenvironment{usecase}{\tabularx{\textwidth}{|0{wl{3cm}}|0{X}|}\hline}{\endtabularx}
\setlength{\parindent}{0em}
\setlength{\parskip}{1em}

\paragraph{\underline{FR-1.1 Rejestracja użytkownika}}\mbox{}\\[1mm]
	\noindent\textbf{Opis:} Rejestracja użytkownika ze standardową rolą, przy użyciu maila, nazwy użytkownika, imienia, nazwiska, hasła i potwierdzenia hasła.\\
	\noindent\textbf{Aktorzy:} Pracownik\\
	\textbf{Warunki początkowe:} Brak\\
	\textbf{Przebieg podstawowy:}
	\begin{enumerate}[noparskip]
		\item Podanie w formularzu maila, nazwy użytkownika, Imienia, nazwiska, hasła i potwierdzenia hasła
    \item Kliknięcie przycisku zarejestruj
		\item Przekierowanie do okna logowania
	\end{enumerate}
	\textbf{Przebieg alternatywny 1 (odrzucenie próby logowania):}
	\begin{enumerate}[noparskip]\setcounter{enumi}{4}
		\item Użytkownik wprowadził 2 różne hasła w polach hasło i powtórz hasło przez co rejestracja nie powiodła się
		\item Użytkownik o zadanym mailu lub nazwie użytkownika istnieje przez co rejestracja nie powiodła się
	\end{enumerate}	
%%  \textbf{Warunki końcowe:} Brak. %\mbox{}\\[-4mm]

\paragraph{\underline{FR-1.2 Logowanie użytkownika}}\mbox{}\\[1mm]
	\noindent\textbf{Opis:} Logowanie użytkownika przy pomocy email lub nazwy użytkownika oraz hasła.\\
	\noindent\textbf{Aktorzy:} Pracownik, Administrator\\
	\textbf{Warunki początkowe:} Użytkownik został wcześniej zarejestrowany w systemie\\
	\textbf{Przebieg podstawowy:}
	\begin{enumerate}[noparskip]
		\item Podanie w formularzu nazwy użytkownika lub email oraz podanie hasła
    \item Kliknięcie przycisku zaloguj
		\item Nadanie użytkownikowi odpowiednich uprawnień
		\item Przekierowanie na stronę domową
	\end{enumerate} 
	\indent \textbf{Przebieg alternatywny 1 (odrzucenie próby logowania):}
	\begin{enumerate}[noparskip]\setcounter{enumi}{3}
		\item Podane dane w formularzu są nieprawidłowe, przez co nie następuje logowanie i pozostaje na stronie logowania
	\end{enumerate}	

\paragraph{\underline{FR-1.3 Usuwanie pracownika}}\mbox{}\\[1mm]
	\noindent\textbf{Opis:}	Administrator jako zarządca systemu może usunąć użytkownika.\\
	\noindent\textbf{Aktorzy:} Administrator\\
	\textbf{Warunki początkowe:} Administrator został wcześniej zarejestrowany w systemie i posiada uprawnienia administratora\\
	\textbf{Include:} 
	\begin{enumerate}[noparskip]
		\item FR-1.5 Przeglądanie zarejestrowanych pracowników	
	\end{enumerate}
  \textbf{Przebieg podstawowy:}
	\begin{enumerate}[noparskip]
		\item Kliknięcie przycisku usuń na wierszu reprezentującym pracownika
    \item Usunięcie z bazy danych pracownika oraz jego historii uczestnictwa w loteriach
	\end{enumerate}

\paragraph{\underline{FR-1.4 Wylogowywanie użytkownika}}\mbox{}\\[1mm]
	\noindent\textbf{Opis:} Wylogowywanie z systemu\\
	\noindent\textbf{Aktorzy:} Pracownik, Administrator\\
	\textbf{Warunki początkowe:} Użytkownik został wcześniej zalogowany\\
	\textbf{Przebieg podstawowy:}
	\begin{enumerate}[noparskip]
		\item Kliknięcie przycisku wyloguj
		\item Przejście na stronę logowania
	\end{enumerate} 

\paragraph{\underline{FR-1.5 Przeglądanie zarejestrowanych pracowników}}\mbox{}\\[1mm]
	\noindent\textbf{Opis:} Możliwe jest wyświetlanie wszystkich użytkowników z rolą Pracownik\\
	\noindent\textbf{Aktorzy:} Administrator\\
	\textbf{Warunki początkowe:} Administrator został wcześniej zarejestrowany w systemie i posiada uprawnienia administratora\\
	\textbf{Przebieg podstawowy:}
	\begin{enumerate}[noparskip]
		\item Kliknięcie przycisku zarządzaj użytkownikami
		\item Wyświetlenie widoku użytkowników
	\end{enumerate} 


\paragraph{\underline{FR-2.1 Przeglądanie sprzętu komputerowego}}\mbox{}\\[1mm]
	\noindent\textbf{Opis:} Administrator widzi wszystkie sprzęty dodane w systemie, natomiast pracownik tylko te któe są gotowe do loterii\\
	\noindent\textbf{Aktorzy:} Pracownik, Administrator\\
	\textbf{Warunki początkowe:} Użytkownik jest zalogowany w systemie\\
	\textbf{Extend:} 
	\begin{enumerate}[noparskip]
		\item Filtrowanie jaki rodzaj sprzętu ma być wyświetlany na widoku lub wybranie wszystkich rodzajów
		\item Filtrowanie sprzętu po biurze w jakim się znajdują
	\end{enumerate}
	\textbf{Przebieg podstawowy:}
	\begin{enumerate}[noparskip]
		\item Wyświetlenie widoku sprzętów komputerowych na stronie domowej, z dostępnymi operacjami zarządzania sprzętem i loteriami tych sprzętów
	\end{enumerate} 
	\textbf{Przebieg alternatywny:}
	\begin{enumerate}[noparskip]
		\item[1b] Użytkownik o dostępie pracownika nie ma dostępu do operacji zarządzania sprzętem
	\end{enumerate} 

\paragraph{\underline{FR-2.2 Dodanie nowego sprzętu komputerowego}}\mbox{}\\[1mm]
	\noindent\textbf{Opis:} Administrator może dodawać 3 różne rodzaje urządzeń zdefiniowanych w systemie\\
	\noindent\textbf{Aktorzy:} Administrator\\
	\textbf{Warunki początkowe:} Administrator został wcześniej zarejestrowany w systemie i posiada uprawnienia administratora\\
	\textbf{Extend:} 
	\begin{enumerate}[noparskip]
		\item FR-3.2 Dodawanie komponentu dla komputera
	\end{enumerate}
	\textbf{Przebieg podstawowy:}
	\begin{enumerate}[noparskip]
		\item Kliknięcie przycisku odpowiedzialnego za dodanie urządzenia
    \item Wybranie z listy rozwijanej typu urządzenia które będzie dodawane
		\item System wyświetla formularz z odpowiednim rodzajem urządzenia
		\item Wpisanie oraz wybranie odpowiednich parametrów reprezentujących urządzenie
		\item Dodanie urządzenia do systemu	\end{enumerate} 
	\textbf{Przebieg alternatywny:}
	\begin{enumerate}[noparskip]\setcounter{enumi}{3}
		\item Administrator nie znalazł w formularzu komputera; procesora, ram, lub dysku i postanowił dodać taki klikając odpowiedni przycisk: dodaj procesor, dodaj ram lub dodaj pamięć
		\item Następuje teraz wykonanie przypadku FR-3.2 Dodawanie komponentu dla komputera
		\item Kontynuacja uzupełniania formularza
		\item Dodanie nowego komputera
	\end{enumerate} 

\paragraph{\underline{FR-2.3 Modyfikacja sprzętu komputerowego}}\mbox{}\\[1mm]
	\noindent\textbf{Opis:} Możliwa jest korekta urządzeń lub podanie brakujących parametrów urządzenia\\
	\noindent\textbf{Aktorzy:} Administrator\\
	\textbf{Warunki początkowe:} Administrator został wcześniej zarejestrowany w systemie i posiada uprawnienia administratora\\
	\textbf{Include:} 
	\begin{enumerate}[noparskip]
		\item FR-2.1 Przeglądanie sprzętu komputerowego
	\end{enumerate}
	\textbf{Przebieg podstawowy:}
	\begin{enumerate}[noparskip]
    \item Kliknięcie przycisku modyfikuj na odpowiednim wierszu reprezentującym urządzenie
	  \item System wykrywa odpowiedni rodzaj urządzenia i wyświetla formularz z poprzednimi danymi urządzenia
	  \item Wpisanie oraz wybranie odpowiednich parametrów reprezentujących urządzenie
	  \item Modyfikacja urządzenia
	\end{enumerate} 
	\textbf{Przebieg alternatywny:}
	\begin{enumerate}[noparskip]\setcounter{enumi}{4}
		\item Administrator nie znalazł w formularzu komputera; procesora, ram, lub dysku i postanowił dodać taki klikając odpowiedni przycisk: dodaj procesor, dodaj ram lub dodaj pamięć
		\item Następuje teraz wykonanie przypadku FR-3.2 Dodawanie komponentu dla komputera
		\item Kontynuacja uzupełniania formularza
		\item Modyfikacja komputera
	\end{enumerate} 
\paragraph{\underline{FR-2.4 Usuwanie sprzętu komputerowego}}\mbox{}\\[1mm]
	\noindent\textbf{Opis:} Administrator może usuwać urządzenia\\
	\noindent\textbf{Aktorzy:} Administrator\\
	\textbf{Warunki początkowe:} Administrator został wcześniej zarejestrowany w systemie i posiada uprawnienia administratora\\
	\textbf{Include:} 
	\begin{enumerate}[noparskip]
		\item FR-2.1 Przeglądanie sprzętu komputerowego
	\end{enumerate}
    \textbf{Przebieg podstawowy:}
	\begin{enumerate}[noparskip]
		\item Kliknięcie przycisku usuń na odpowiednim wierszu reprezentującym urządzenie
		\item Usunięcie sprzętu z systemu
    \end{enumerate}

\paragraph{\underline{FR-2.5 Wyświetlanie szczegółowych informacji dotyczących sprzętu}}\mbox{}\\[1mm]
	\noindent\textbf{Opis:} Możliwe jest szczegółowe sprawdzenie dotyczące parametrów sprzętów w specjalnym formularzu\\
	\noindent\textbf{Aktorzy:} Administrator, Pracownik\\
	\textbf{Warunki początkowe:} Użytkownik został zalogowany do systemu\\
	\textbf{Include:} 
	\begin{enumerate}[noparskip]
		\item FR-2.1 Przeglądanie sprzętu komputerowego
	\end{enumerate}
    \textbf{Przebieg podstawowy:}
	\begin{enumerate}[noparskip]
		\item Kliknięcie przycisku informacji o urządzeniu na odpowiednim wierszu reprezentującym urządzenie
		\item System wykrywa odpowiedni rodzaj urządzenia i wyświetla formularz urządzenia z zablokowanymi polami 
    \end{enumerate}

\paragraph{\underline{FR-3.1 Przeglądanie komponentów komputera}}\mbox{}\\[1mm]
	\noindent\textbf{Opis:} Do lepszego oszacowania ceny sprzętu Administrator ma widok komponentów które może posiadać komputer\\
	\noindent\textbf{Aktorzy:} Administrator\\
	\textbf{Warunki początkowe:} Administrator został wcześniej zarejestrowany w systemie i posiada uprawnienia administratora\\
	\textbf{Extend:}
    \begin{enumerate}[noparskip]
		\item filtrowanie komponentów po rodzaju: procesor, ram, dysk pamięci
	\end{enumerate}
    \textbf{Przebieg podstawowy:}
	\begin{enumerate}[noparskip]
		\item Kliknięcie przycisku odpowiedzialnego za widok komponentów
		\item Wyświetlenie widoku na stronie
    \end{enumerate}

\paragraph{\underline{FR-3.2 Dodawanie nowego komponentu komputera}}\mbox{}\\[1mm]
	\noindent\textbf{Opis:} Możliwe jest dodawanie komponentu na który składa się cena i nazwa\\
	\noindent\textbf{Aktorzy:} Administrator\\
	\textbf{Warunki początkowe:} Administrator został wcześniej zarejestrowany w systemie i posiada uprawnienia administratora\\
    \textbf{Przebieg podstawowy:}
    \begin{enumerate}[noparskip]
		\item Wykonanie przypadku FR-3.1 Przeglądanie komponentów(procesory, ram, dyski) komputera
        \item Kliknięcie przycisku dodaj procesor, dodaj RAM lub dodaj dysk
		\item System wyświetla formularz komponentu
		\item Wpisanie nazwy i ceny komponentu
		\item Dodanie komponentu do bazy danych
    \end{enumerate}
    \textbf{Przebieg podstawowy:}
    \begin{enumerate}[noparskip]
		\item[1b] Administrator wykonuje dodanie z poziomu formularza dodawania komputera
		\item[4] Administrator podał nazwę komponentu która istnieje w bazie danych
		\item[5] Następuje modyfikacja podanej ceny istniejącego komponentu zamiast dodania nowego
    \end{enumerate}

\paragraph{\underline{FR-3.3 Modyfikacja komponentu komputera}}\mbox{}\\[1mm]
	\noindent\textbf{Opis:} Możliwe jest modyfikowanie komponentu na który składa się cena i nazwa\\
	\noindent\textbf{Aktorzy:} Administrator\\
	\textbf{Warunki początkowe:} Administrator został wcześniej zarejestrowany w systemie i posiada uprawnienia administratora\\
	\textbf{Include:} 
	\begin{enumerate}[noparskip]
		\item FR-3.1 Przeglądanie komponentów(procesory, ram, dyski) komputera
	\end{enumerate}
    \textbf{Przebieg podstawowy:}
	\begin{enumerate}[noparskip]
		\item Kliknięcie przycisku modyfikuj na odpowiednim wierszu reprezentującym komponent
		\item Modyfikacja komponentu w bazie danych
    \end{enumerate}

\paragraph{\underline{FR-3.4 Usuwanie komponentu komputera}}\mbox{}\\[1mm]
	\noindent\textbf{Opis:} Możliwe jest usuwanie komponentów wchodzącym w skład komputera\\
	\noindent\textbf{Aktorzy:} Administrator\\
	\textbf{Warunki początkowe:} Administrator został wcześniej zarejestrowany w systemie i posiada uprawnienia administratora\\
	\textbf{Include:} 
	\begin{enumerate}[noparskip]
		\item FR-3.1 Przeglądanie komponentów(procesory, ram, dyski) komputera
	\end{enumerate}
    \textbf{Przebieg podstawowy:}
	\begin{enumerate}[noparskip]
		\item Kliknięcie przycisku usuń na odpowiednim wierszu reprezentującym komponent
		\item Usunięcie komponentu z bazy, oraz ustawienie wszystkich odwołań w komputerach do danego komponentu na null
    \end{enumerate}

\paragraph{\underline{FR-4.1 Zmiana gotowości sprzętu do losowania}}\mbox{}\\[1mm]
	\noindent\textbf{Opis:} Administrator zarządza urządzeniem i określa jego gotowość do loterii\\
	\noindent\textbf{Aktorzy:} Administrator\\
	\textbf{Warunki początkowe:} Administrator został wcześniej zarejestrowany w systemie i posiada uprawnienia administratora, Urządzenie nie zostało wcześniej wylosowane w loterii\\
	\textbf{Include:} 
	\begin{enumerate}[noparskip]
		\item FR-2.1 Przeglądanie sprzętu komputerowego
	\end{enumerate}
    \textbf{Przebieg podstawowy:}
	\begin{enumerate}[noparskip]
		\item Kliknięcie przycisku gotowości do losowania na odpowiednim wierszu reprezentującym urządzenie
		\item Zmiana gotowości losowania na przeciwny do poprzedniego
    \end{enumerate}

\paragraph{\underline{FR-4.2 Zmiana statusu sprzętu na sprzedane lub niesprzedane}}\mbox{}\\[1mm]
	\noindent\textbf{Opis:} Administrator kontroluje które sprzęty zostały już sprzedane i dostarczone do pracownika\\
	\noindent\textbf{Aktorzy:} Administrator\\
	\textbf{Warunki początkowe:} Administrator został wcześniej zarejestrowany w systemie i posiada uprawnienia administratora, Urządzenie zostało wylosowane w loterii\\
	\textbf{Include:} 
	\begin{enumerate}[noparskip]
		\item FR-2.1 Przeglądanie sprzętu komputerowego
	\end{enumerate}
    \textbf{Przebieg podstawowy:}
	\begin{enumerate}[noparskip]
		\item Kliknięcie przycisku odpowiedzialnego za zmianę statusu sprzedane na odpowiednim wierszu reprezentującym urządzenie
		\item Aktualizacja statusu sprzedane urządzenia w systemie
    \end{enumerate}

\paragraph{\underline{FR-4.3 Wyświetlanie listy uczestników biorącym udział w losowaniu}}\mbox{}\\[1mm]
	\noindent\textbf{Opis:} Administrator sprawdza jacy użytkownicy biorą udział w losowaniu sprzętu\\
	\noindent\textbf{Aktorzy:} Administrator\\
	\textbf{Warunki początkowe:} Administrator został wcześniej zarejestrowany w systemie i posiada uprawnienia administratora, Status urządzenia został ustawiony na gotowe do losowania\\
	\textbf{Include:} 
	\begin{enumerate}[noparskip]
		\item FR-2.1 Przeglądanie sprzętu komputerowego
	\end{enumerate}
    \textbf{Przebieg podstawowy:}
	\begin{enumerate}[noparskip]
		\item Kliknięcie przycisku odpowiedzialnego za wyświetlanie listy uczestników na odpowiednim wierszu reprezentującym urządzenie
		\item Przekierowanie do widoku użytkowników biorącym udział w losowaniu wybranego sprzętu
		\item Wyświetlanie widoku uczestników losowania, oraz ich statusu dotyczącego czy już wygrali loterię
    \end{enumerate}

\paragraph{\underline{FR-4.4 Losowanie zwycięzcy loterii}}\mbox{}\\[1mm]
	\noindent\textbf{Opis:} Administrator losuje zwycięzce spośród tych pracowników którzy zapisali się a loterię\\
	\noindent\textbf{Aktorzy:} Administrator\\
	\textbf{Warunki początkowe:} Administrator został wcześniej zarejestrowany w systemie i posiada uprawnienia administratora, Status urządzenia został ustawiony na gotowe do losowania, Przynajmniej jeden pracownik jest zapisany na losowanie\\
	\textbf{Include:} 
	\begin{enumerate}[noparskip]
		\item FR-2.1 Przeglądanie sprzętu komputerowego
	\end{enumerate}
    \textbf{Przebieg podstawowy:}
	\begin{enumerate}[noparskip]
		\item Kliknięcie przycisku odpowiedzialnego za losowanie zwycięzcy na odpowiednim wierszu reprezentującym urządzenie
		\item Ustawienie daty losowania na dzisiejszą
		\item Ustawienie wylosowanego pracownika jako zwycięzce losowania 
    \end{enumerate}

\paragraph{\underline{FR-5.1 Przeglądanie historii loterii wybranego użytkownika}}\mbox{}\\[1mm]
	\noindent\textbf{Opis:} W systemie istnieje filtrowanie dotyczących historii loterii\\
	\noindent\textbf{Aktorzy:} Administrator, Pracownik\\
	\textbf{Warunki początkowe:} Użytkownik został zalogowany w systemie\\
	\textbf{Extend:} 
	\begin{enumerate}[noparskip]
		\item Filtrowanie loterii po statusie uczestnika: wygrana, przegrana, trwająca
		\item FR-2.5 Wyświetlanie szczegółowych informacji dotyczących sprzętu komputerowego
	\end{enumerate}
    \textbf{Przebieg podstawowy:}
	\begin{enumerate}[noparskip]
		\item Kliknięcie przycisku pokaż historie na odpowiednim wierszu reprezentującym użytkownika
		\item Wyświetlenie sprzętu, daty loterii oraz statusu użytkownika w odniesieniu do bieżącej loterii 
    \end{enumerate}
    \textbf{Przebieg alternatywny:}
	\begin{enumerate}[noparskip]
		\item [1b] Kliknięcie przycisku pokaż historie użytkownika który jest właśnie zalogowany
	\end{enumerate}

\paragraph{\underline{FR-5.2 Zapisywanie się na loterię urządzenia}}\mbox{}\\[1mm]
	\noindent\textbf{Opis:} Pracownik przeglądając specyfikacje sprzętu może być zainteresowany wzięciem udziału w loterii\\
	\noindent\textbf{Aktorzy:} Pracownik\\
	\textbf{Warunki początkowe:} Pracownik został wcześniej zalogowany w systemie i posiada uprawnienia pracownika, sprzęt nie został wcześniej wylosowany, pracownik wcześniej nie zapisał się na loterię dotyczącego tego sprzętu\\
	\textbf{Include:} 
	\begin{enumerate}[noparskip]
		\item FR-2.1 Przeglądanie sprzętu komputerowego
	\end{enumerate}
    \textbf{Przebieg podstawowy:}
	\begin{enumerate}[noparskip]
		\item Pracownik klika przycisk weź udział na odpowiednim wierszu reprezentującym urządzenie
		\item Do systemu zostaje dodane uczestnictwo w loterii wybranego urządzenia
    \end{enumerate}

\paragraph{\underline{FR-5.3 Wypisanie się z loterii}}\mbox{}\\[1mm]
	\noindent\textbf{Opis:} Pracownik może chcieć się wypisać z loterii\\
	\noindent\textbf{Aktorzy:} Pracownik\\
	\textbf{Warunki początkowe:} Pracownik został wcześniej zalogowany w systemie i posiada uprawnienia pracownika, sprzęt nie został wcześniej wylosowany, pracownik wcześniej zapisał się na loterię dotyczącego wybranego sprzętu\\
	\textbf{Include:} 
	\begin{enumerate}[noparskip]
		\item FR-2.1 Przeglądanie sprzętu komputerowego
	\end{enumerate}
    \textbf{Przebieg podstawowy:}
	\begin{enumerate}[noparskip]
		\item Pracownik klika przycisk wypisz się na odpowiednim wierszu reprezentującym urządzenie
		\item Z systemu zostaje usunięte uczestnictwo w loterii wybranego urządzenia
    \end{enumerate}

\section{Wymagania niefunkcjonalne}

%TODO










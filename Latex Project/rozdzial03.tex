\chapter{Analiza wymagań}

\section{Wymagania funkcjonalne}

Projektowany system powinien zapewnieć szereg funkcji do potencjalnego wykorzystania przez jego użytkowników. Poniżej wypisano te funkcje w podziale z uwagi na ich zastosowanie. Dodatkowo dostarczono opis podstawowych przypadków ich użycia.
\begin{enumerate}[label={\textbf{FR}-\bfseries\arabic*}]
    \item \textbf{Zarządzanie użytkownikami:}
    \begin{enumerate}[label={FR-\arabic{enumi}.\arabic*},noparskip]
        \item rejestracja użytkownika,
        \item logowanie użytkownika,
        \item usuwanie pracownika,
        \item wylogowywanie użytkownika,
				\item przeglądanie zarejestrowanych pracowników.
    \end{enumerate}
		
		\item \textbf{Zarządzanie sprzętem komputerowym:}
    \begin{enumerate}[label={FR-\arabic{enumi}.\arabic*},noparskip]
        \item przeglądanie sprzętu komputerowego,
        \item dodawanie nowego sprzętu komputerowego,
        \item modyfikacja sprzętu komputerowego,
        \item usuwanie sprzętu komputerowego,
				\item wyświetlanie szczegółowych informacji dotyczących sprzętu komputerowego.
		\end{enumerate}
		
		\item \textbf{Zarządzanie komponentami komputera:}
			\begin{enumerate}[label={FR-\arabic{enumi}.\arabic*},noparskip]		
				\item przeglądanie komponentów komputera,
				\item dodawanie komponentu(procesora, ram, dysku) dla komputera,
				\item modyfikacja komponentu komputera,
				\item usuwanie komponentu komputera.
    \end{enumerate}
		
		
		\item \textbf{Zarządzanie loterią wybranego sprzętu komputerowego}
		\begin{enumerate}[label={FR-\arabic{enumi}.\arabic*},noparskip]
        \item zmiana gotowości sprzętu do losowania,
        \item zmiana statusu sprzętu na sprzedane lub niesprzedane,
				\item wyświetlanie listy uczestników biorących udział w losowaniu,
        \item losowanie zwycięzcy loterii.
    \end{enumerate}
		
		\item \textbf{Zarządzanie uczestnictwem w loterii}
			\begin{enumerate}[label={FR-\arabic{enumi}.\arabic*},noparskip]
        \item przeglądanie historii loterii wybranego użytkownika,
        \item zapisywanie się na loterię urządzenia,
				\item wypisywanie się z loterii urządzenia.
    \end{enumerate}
\end{enumerate}

\paragraph{\underline{FR-1.1 Rejestracja użytkownika}}\mbox{}\\[1mm]
	\noindent\textbf{Opis:} Rejestracja użytkownika ze standardową rolą, przy użyciu maila, nazwy użytkownika, imienia, nazwiska, hasła i potwierdzenia hasła.\\
	\noindent\textbf{Aktorzy:} Pracownik\\
	\textbf{Warunki początkowe:} Brak.\\
	\textbf{Przebieg podstawowy:}
	\begin{enumerate}[noparskip]
		\item Podanie w formularzu maila, nazwy użytkownika, Imienia, nazwiska, hasła i potwierdzenia hasła.
    \item Kliknięcie przycisku zarejestruj.
		\item Przekierowanie do okna logowania.
	\end{enumerate}
	\textbf{Przebieg alternatywny 1 (odrzucenie próby logowania):}
	\begin{enumerate}[noparskip]\setcounter{enumi}{4}
		\item Użytkownik wprowadził 2 różne hasła w polach hasło i powtórz hasło przez co rejestracja nie powiodła się.
		\item Użytkownik o zadanym mailu lub nazwie użytkownika istnieje przez co rejestracja nie powiodła się.
	\end{enumerate}	\mbox{}\\[-11mm]
%%  \textbf{Warunki końcowe:} Brak. %\mbox{}\\[-4mm]

\paragraph{\underline{FR-1.2 Logowanie użytkownika}}\mbox{}\\[1mm]
	\noindent\textbf{Opis:} Logowanie użytkownika przy pomocy email lub nazwy użytkownika oraz hasła.\\
	\noindent\textbf{Aktorzy:} Pracownik, Administrator\\
	\textbf{Warunki początkowe:} Użytkownik został wcześniej zarejestrowany w systemie.\\
	\textbf{Przebieg podstawowy:}
	\begin{enumerate}[noparskip]
		\item Podanie w formularzu nazwy użytkownika lub email oraz podanie hasła.
    \item Kliknięcie przycisku zaloguj.
		\item Nadanie użytkownikowi odpowiednich uprawnień.
		\item Przekierowanie na stronę domową.
	\end{enumerate} 
	\indent \textbf{Przebieg alternatywny 1 (odrzucenie próby logowania):}
	\begin{enumerate}[noparskip]\setcounter{enumi}{3}
		\item Podane dane w formularzu są nieprawidłowe, przez co nie następuje logowanie i pozostaje na stronie logowania.
	\end{enumerate}	\mbox{}\\[-11mm]

\paragraph{\underline{FR-1.3 Usuwanie pracownika}}\mbox{}\\[1mm]
	\noindent\textbf{Opis:}	Administrator jako zarządca systemu może usunąć użytkownika.\\
	\noindent\textbf{Aktorzy:} Administrator\\
	\textbf{Warunki początkowe:} Administrator został wcześniej zarejestrowany w systemie i posiada uprawnienia administratora.\\
	\textbf{Include:} 
	\begin{enumerate}[noparskip]
		\item FR-1.5 Przeglądanie zarejestrowanych pracowników.
	\end{enumerate}
  \textbf{Przebieg podstawowy:}
	\begin{enumerate}[noparskip]
		\item Kliknięcie przycisku usuń na wierszu reprezentującym pracownika.
    \item Usunięcie z bazy danych pracownika oraz jego historii uczestnictwa w loteriach.
	\end{enumerate} \mbox{}\\[-11mm]

\paragraph{\underline{FR-1.4 Wylogowywanie użytkownika.}}\mbox{}\\[1mm]
	\noindent\textbf{Opis:} Wylogowywanie z systemu\\
	\noindent\textbf{Aktorzy:} Pracownik, Administrator\\
	\textbf{Warunki początkowe:} Użytkownik został wcześniej zalogowany.\\
	\textbf{Przebieg podstawowy:}
	\begin{enumerate}[noparskip]
		\item Kliknięcie przycisku wyloguj.
		\item Przejście na stronę logowania.
	\end{enumerate} \mbox{}\\[-11mm]

\paragraph{\underline{FR-1.5 Przeglądanie zarejestrowanych pracowników}}\mbox{}\\[1mm]
	\noindent\textbf{Opis:} Możliwe jest wyświetlanie wszystkich użytkowników z rolą Pracownik.\\
	\noindent\textbf{Aktorzy:} Administrator\\
	\textbf{Warunki początkowe:} Administrator został wcześniej zarejestrowany w systemie i posiada uprawnienia administratora.\\
	\textbf{Przebieg podstawowy:}
	\begin{enumerate}[noparskip]
		\item Kliknięcie przycisku zarządzaj użytkownikami.
		\item Wyświetlenie widoku użytkowników.
	\end{enumerate} \mbox{}\\[-11mm]

\paragraph{\underline{FR-2.1 Przeglądanie sprzętu komputerowego}}\mbox{}\\[1mm]
	\noindent\textbf{Opis:} Administrator widzi wszystkie sprzęty dodane w systemie, natomiast pracownik tylko te które są gotowe do loterii.\\
	\noindent\textbf{Aktorzy:} Pracownik, Administrator\\
	\textbf{Warunki początkowe:} Użytkownik jest zalogowany w systemie.\\
	\textbf{Extend:} 
	\begin{enumerate}[noparskip]
		\item Filtrowanie jaki rodzaj sprzętu ma być wyświetlany na widoku lub wybranie wszystkich rodzajów.
		\item Filtrowanie sprzętu po biurze w jakim się znajdują.
	\end{enumerate}
	\textbf{Przebieg podstawowy:}
	\begin{enumerate}[noparskip]
		\item Wyświetlenie widoku sprzętów komputerowych na stronie domowej, z dostępnymi operacjami zarządzania sprzętem i loteriami tych sprzętów.
	\end{enumerate} 
	\textbf{Przebieg alternatywny:}
	\begin{enumerate}[noparskip]
		\item[1b] Użytkownik o dostępie pracownika nie ma dostępu do operacji zarządzania sprzętem.
	\end{enumerate} \mbox{}\\[-11mm]

\paragraph{\underline{FR-2.2 Dodanie nowego sprzętu komputerowego}}\mbox{}\\[1mm]
	\noindent\textbf{Opis:} Administrator może dodawać 3 różne rodzaje urządzeń zdefiniowanych w systemie.\\
	\noindent\textbf{Aktorzy:} Administrator\\
	\textbf{Warunki początkowe:} Administrator został wcześniej zarejestrowany w systemie i posiada uprawnienia administratora.\\
	\textbf{Extend:} 
	\begin{enumerate}[noparskip]
		\item FR-3.2 Dodawanie komponentu dla komputera.
	\end{enumerate}
	\textbf{Przebieg podstawowy:}
	\begin{enumerate}[noparskip]
		\item Kliknięcie przycisku odpowiedzialnego za dodanie urządzenia.
    \item Wybranie z listy rozwijanej typu urządzenia które będzie dodawane.
		\item System wyświetla formularz z odpowiednim rodzajem urządzenia.
		\item Wpisanie oraz wybranie odpowiednich parametrów reprezentujących urządzenie.
		\item Dodanie urządzenia do systemu.	\end{enumerate} 
	\textbf{Przebieg alternatywny:}
	\begin{enumerate}[noparskip]\setcounter{enumi}{3}
		\item Administrator nie znalazł w formularzu komputera; procesora, ram, lub dysku i postanowił dodać taki klikając odpowiedni przycisk: dodaj procesor, dodaj ram lub dodaj pamięć.
		\item Następuje teraz wykonanie przypadku FR-3.2 Dodawanie komponentu dla komputera.
		\item Kontynuacja uzupełniania formularza.
		\item Dodanie nowego komputera.
	\end{enumerate} \mbox{}\\[-11mm]

\paragraph{\underline{FR-2.3 Modyfikacja sprzętu komputerowego}}\mbox{}\\[1mm]
	\noindent\textbf{Opis:} Możliwa jest korekta urządzeń lub podanie brakujących parametrów urządzenia.\\
	\noindent\textbf{Aktorzy:} Administrator\\
	\textbf{Warunki początkowe:} Administrator został wcześniej zarejestrowany w systemie i posiada uprawnienia administratora.\\
	\textbf{Include:} 
	\begin{enumerate}[noparskip]
		\item FR-2.1 Przeglądanie sprzętu komputerowego.
	\end{enumerate}
	\textbf{Przebieg podstawowy:}
	\begin{enumerate}[noparskip]
    \item Kliknięcie przycisku modyfikuj na odpowiednim wierszu reprezentującym urządzenie.
	  \item System wykrywa odpowiedni rodzaj urządzenia i wyświetla formularz z poprzednimi danymi urządzenia.
	  \item Wpisanie oraz wybranie odpowiednich parametrów reprezentujących urządzenie.
	  \item Modyfikacja urządzenia.
	\end{enumerate} 
	\textbf{Przebieg alternatywny:}
	\begin{enumerate}[noparskip]\setcounter{enumi}{4}
		\item Administrator nie znalazł w formularzu komputera; procesora, ram, lub dysku i postanowił dodać taki klikając odpowiedni przycisk: dodaj procesor, dodaj ram lub dodaj pamięć.
		\item Następuje teraz wykonanie przypadku FR-3.2 Dodawanie komponentu dla komputera.
		\item Kontynuacja uzupełniania formularza.
		\item Modyfikacja komputera.
	\end{enumerate} \mbox{}\\[-11mm]
	
\paragraph{\underline{FR-2.4 Usuwanie sprzętu komputerowego}}\mbox{}\\[1mm]
	\noindent\textbf{Opis:} Administrator może usuwać urządzenia.\\
	\noindent\textbf{Aktorzy:} Administrator\\
	\textbf{Warunki początkowe:} Administrator został wcześniej zarejestrowany w systemie i posiada uprawnienia administratora.\\
	\textbf{Include:} 
	\begin{enumerate}[noparskip]
		\item FR-2.1 Przeglądanie sprzętu komputerowego.
	\end{enumerate}
    \textbf{Przebieg podstawowy:}
	\begin{enumerate}[noparskip]
		\item Kliknięcie przycisku usuń na odpowiednim wierszu reprezentującym urządzenie.
		\item Usunięcie sprzętu z systemu.
  \end{enumerate} \mbox{}\\[-11mm]

\paragraph{\underline{FR-2.5 Wyświetlanie szczegółowych informacji dotyczących sprzętu}}\mbox{}\\[1mm]
	\noindent\textbf{Opis:} Możliwe jest szczegółowe sprawdzenie dotyczące parametrów sprzętów w specjalnym formularzu.\\
	\noindent\textbf{Aktorzy:} Administrator, Pracownik\\
	\textbf{Warunki początkowe:} Użytkownik został zalogowany do systemu.\\
	\textbf{Include:} 
	\begin{enumerate}[noparskip]
		\item FR-2.1 Przeglądanie sprzętu komputerowego.
	\end{enumerate}
  \textbf{Przebieg podstawowy:}
	\begin{enumerate}[noparskip]
		\item Kliknięcie przycisku informacji o urządzeniu na odpowiednim wierszu reprezentującym urządzenie.
		\item System wykrywa odpowiedni rodzaj urządzenia i wyświetla formularz urządzenia z zablokowanymi polami.
  \end{enumerate} \mbox{}\\[-11mm]

\paragraph{\underline{FR-3.1 Przeglądanie komponentów komputera}}\mbox{}\\[1mm]
	\noindent\textbf{Opis:} Do lepszego oszacowania ceny sprzętu Administrator ma widok komponentów które może posiadać komputer.\\
	\noindent\textbf{Aktorzy:} Administrator\\
	\textbf{Warunki początkowe:} Administrator został wcześniej zarejestrowany w systemie i posiada uprawnienia administratora.\\
	\textbf{Extend:}
    \begin{enumerate}[noparskip]
		\item Filtrowanie komponentów po rodzaju: procesor, ram, dysk pamięci.
	\end{enumerate}
  \textbf{Przebieg podstawowy:}
	\begin{enumerate}[noparskip]
		\item Kliknięcie przycisku odpowiedzialnego za widok komponentów.
		\item Wyświetlenie widoku na stronie.
  \end{enumerate} \mbox{}\\[-11mm]

\paragraph{\underline{FR-3.2 Dodawanie nowego komponentu komputera}}\mbox{}\\[1mm]
	\noindent\textbf{Opis:} Możliwe jest dodawanie komponentu na który składa się cena i nazwa.\\
	\noindent\textbf{Aktorzy:} Administrator\\
	\textbf{Warunki początkowe:} Administrator został wcześniej zarejestrowany w systemie i posiada uprawnienia administratora.\\
  \textbf{Przebieg podstawowy:}
  \begin{enumerate}[noparskip]
		\item Wykonanie przypadku FR-3.1 Przeglądanie komponentów(procesory, ram, dyski) komputera.
		\item Kliknięcie przycisku dodaj procesor, dodaj RAM lub dodaj dysk.
		\item System wyświetla formularz komponentu.
		\item Wpisanie nazwy i ceny komponentu.
		\item Dodanie komponentu do bazy danych.
  \end{enumerate}
  \textbf{Przebieg podstawowy:}
  \begin{enumerate}[noparskip]
		\item[1b] Administrator wykonuje dodanie z poziomu formularza dodawania komputera.
		\item[4] Administrator podał nazwę komponentu która istnieje w bazie danych.
		\item[5] Następuje modyfikacja podanej ceny istniejącego komponentu zamiast dodania nowego.
  \end{enumerate} \mbox{}\\[-11mm]

\paragraph{\underline{FR-3.3 Modyfikacja komponentu komputera}}\mbox{}\\[1mm]
	\noindent\textbf{Opis:} Możliwe jest modyfikowanie komponentu na który składa się cena i nazwa.\\
	\noindent\textbf{Aktorzy:} Administrator\\
	\textbf{Warunki początkowe:} Administrator został wcześniej zarejestrowany w systemie i posiada uprawnienia administratora.\\
	\textbf{Include:} 
	\begin{enumerate}[noparskip]
		\item FR-3.1 Przeglądanie komponentów(procesory, ram, dyski) komputera.
	\end{enumerate}
  \textbf{Przebieg podstawowy:}
	\begin{enumerate}[noparskip]
		\item Kliknięcie przycisku modyfikuj na odpowiednim wierszu reprezentującym komponent.
		\item Modyfikacja komponentu w bazie danych.
  \end{enumerate} \mbox{}\\[-11mm]

\paragraph{\underline{FR-3.4 Usuwanie komponentu komputera}}\mbox{}\\[1mm]
	\noindent\textbf{Opis:} Możliwe jest usuwanie komponentów wchodzącym w skład komputera.\\
	\noindent\textbf{Aktorzy:} Administrator\\
	\textbf{Warunki początkowe:} Administrator został wcześniej zarejestrowany w systemie i posiada uprawnienia administratora.\\
	\textbf{Include:} 
	\begin{enumerate}[noparskip]
		\item FR-3.1 Przeglądanie komponentów(procesory, ram, dyski) komputera.
	\end{enumerate}
    \textbf{Przebieg podstawowy:}
	\begin{enumerate}[noparskip]
		\item Kliknięcie przycisku usuń na odpowiednim wierszu reprezentującym komponent.
		\item Usunięcie komponentu z bazy, oraz ustawienie wszystkich odwołań w komputerach do danego komponentu na null.
  \end{enumerate} \mbox{}\\[-11mm]

\paragraph{\underline{FR-4.1 Zmiana gotowości sprzętu do losowania}}\mbox{}\\[1mm]
	\noindent\textbf{Opis:} Administrator zarządza urządzeniem i określa jego gotowość do loterii.\\
	\noindent\textbf{Aktorzy:} Administrator\\
	\textbf{Warunki początkowe:} Administrator został wcześniej zarejestrowany w systemie i posiada uprawnienia administratora, Urządzenie nie zostało wcześniej wylosowane w loterii.\\
	\textbf{Include:} 
	\begin{enumerate}[noparskip]
		\item FR-2.1 Przeglądanie sprzętu komputerowego.
	\end{enumerate}
    \textbf{Przebieg podstawowy:}
	\begin{enumerate}[noparskip]
		\item Kliknięcie przycisku gotowości do losowania na odpowiednim wierszu reprezentującym urządzenie.
		\item Zmiana gotowości losowania na przeciwny do poprzedniego.
  \end{enumerate} \mbox{}\\[-11mm]

\paragraph{\underline{FR-4.2 Zmiana statusu sprzętu na sprzedane lub niesprzedane}}\mbox{}\\[1mm]
	\noindent\textbf{Opis:} Administrator kontroluje które sprzęty zostały już sprzedane i dostarczone do pracownika\\
	\noindent\textbf{Aktorzy:} Administrator\\
	\textbf{Warunki początkowe:} Administrator został wcześniej zarejestrowany w systemie i posiada uprawnienia administratora, Urządzenie zostało wylosowane w loterii.\\
	\textbf{Include:} 
	\begin{enumerate}[noparskip]
		\item FR-2.1 Przeglądanie sprzętu komputerowego
	\end{enumerate}
  \textbf{Przebieg podstawowy:}
	\begin{enumerate}[noparskip]
		\item Kliknięcie przycisku odpowiedzialnego za zmianę statusu sprzedane na odpowiednim wierszu reprezentującym urządzenie.
		\item Aktualizacja statusu sprzedane urządzenia w systemie.
  \end{enumerate} \mbox{}\\[-11mm]

\paragraph{\underline{FR-4.3 Wyświetlanie listy uczestników biorącym udział w losowaniu}}\mbox{}\\[1mm]
	\noindent\textbf{Opis:} Administrator sprawdza jacy użytkownicy biorą udział w losowaniu sprzętu.\\
	\noindent\textbf{Aktorzy:} Administrator\\
	\textbf{Warunki początkowe:} Administrator został wcześniej zarejestrowany w systemie i posiada uprawnienia administratora, Status urządzenia został ustawiony na gotowe do losowania.\\
	\textbf{Include:} 
	\begin{enumerate}[noparskip]
		\item FR-2.1 Przeglądanie sprzętu komputerowego.
	\end{enumerate}
  \textbf{Przebieg podstawowy:}
	\begin{enumerate}[noparskip]
		\item Kliknięcie przycisku odpowiedzialnego za wyświetlanie listy uczestników na odpowiednim wierszu reprezentującym urządzenie.
		\item Przekierowanie do widoku użytkowników biorącym udział w losowaniu wybranego sprzętu.
		\item Wyświetlanie widoku uczestników losowania, oraz ich statusu dotyczącego czy już wygrali loterię.
  \end{enumerate} \mbox{}\\[-11mm]

\paragraph{\underline{FR-4.4 Losowanie zwycięzcy loterii}}\mbox{}\\[1mm]
	\noindent\textbf{Opis:} Administrator losuje zwycięzce spośród tych pracowników którzy zapisali się a loterię.\\
	\noindent\textbf{Aktorzy:} Administrator\\
	\textbf{Warunki początkowe:} Administrator został wcześniej zarejestrowany w systemie i posiada uprawnienia administratora, Status urządzenia został ustawiony na gotowe do losowania, Przynajmniej jeden pracownik jest zapisany na losowanie.\\
	\textbf{Include:} 
	\begin{enumerate}[noparskip]
		\item FR-2.1 Przeglądanie sprzętu komputerowego.
	\end{enumerate}
  \textbf{Przebieg podstawowy:}
	\begin{enumerate}[noparskip]
		\item Kliknięcie przycisku odpowiedzialnego za losowanie zwycięzcy na odpowiednim wierszu reprezentującym urządzenie.
		\item Ustawienie daty losowania na dzisiejszą.
		\item Ustawienie wylosowanego pracownika jako zwycięzce losowania.
  \end{enumerate} \mbox{}\\[-11mm]

\paragraph{\underline{FR-5.1 Przeglądanie historii loterii wybranego użytkownika}}\mbox{}\\[1mm]
	\noindent\textbf{Opis:} W systemie istnieje filtrowanie dotyczących historii loterii.\\
	\noindent\textbf{Aktorzy:} Administrator, Pracownik\\
	\textbf{Warunki początkowe:} Użytkownik został zalogowany w systemie.\\
	\textbf{Extend:} 
	\begin{enumerate}[noparskip]
		\item Filtrowanie loterii po statusie uczestnika: wygrana, przegrana, trwająca.
		\item FR-2.5 Wyświetlanie szczegółowych informacji dotyczących sprzętu komputerowego.
	\end{enumerate}
    \textbf{Przebieg podstawowy:}
	\begin{enumerate}[noparskip]
		\item Kliknięcie przycisku pokaż historie na odpowiednim wierszu reprezentującym użytkownika.
		\item Wyświetlenie sprzętu, daty loterii oraz statusu użytkownika w odniesieniu do bieżącej loterii .
    \end{enumerate}
    \textbf{Przebieg alternatywny:}
	\begin{enumerate}[noparskip]
		\item [1b] Kliknięcie przycisku pokaż historie użytkownika który jest właśnie zalogowany.
	\end{enumerate} \mbox{}\\[-11mm]

\paragraph{\underline{FR-5.2 Zapisywanie się na loterię urządzenia}}\mbox{}\\[1mm]
	\noindent\textbf{Opis:} Pracownik przeglądając specyfikacje sprzętu może być zainteresowany wzięciem udziału w loterii.\\
	\noindent\textbf{Aktorzy:} Pracownik\\
	\textbf{Warunki początkowe:} Pracownik został wcześniej zalogowany w systemie i posiada uprawnienia pracownika, sprzęt nie został wcześniej wylosowany, pracownik wcześniej nie zapisał się na loterię dotyczącego tego sprzętu.\\
	\textbf{Include:} 
	\begin{enumerate}[noparskip]
		\item FR-2.1 Przeglądanie sprzętu komputerowego.
	\end{enumerate}
  \textbf{Przebieg podstawowy:}
	\begin{enumerate}[noparskip]
		\item Pracownik klika przycisk weź udział na odpowiednim wierszu reprezentującym urządzenie.
		\item Do systemu zostaje dodane uczestnictwo w loterii wybranego urządzenia.
  \end{enumerate} \mbox{}\\[-11mm]

\paragraph{\underline{FR-5.3 Wypisanie się z loterii}}\mbox{}\\[1mm]
	\noindent\textbf{Opis:} Pracownik może chcieć się wypisać z loterii.\\
	\noindent\textbf{Aktorzy:} Pracownik\\
	\textbf{Warunki początkowe:} Pracownik został wcześniej zalogowany w systemie i posiada uprawnienia pracownika, sprzęt nie został wcześniej wylosowany, pracownik wcześniej zapisał się na loterię dotyczącego wybranego sprzętu.\\
	\textbf{Include:} 
	\begin{enumerate}[noparskip]
		\item FR-2.1 Przeglądanie sprzętu komputerowego.
	\end{enumerate}
  \textbf{Przebieg podstawowy:}
	\begin{enumerate}[noparskip]
		\item Pracownik klika przycisk wypisz się na odpowiednim wierszu reprezentującym urządzenie.
		\item Z systemu zostaje usunięte uczestnictwo w loterii wybranego urządzenia.
  \end{enumerate} \mbox{}\\[-11mm]

\section{Wymagania niefunkcjonalne}

Etapem projektowania systemu jest zdefiniowanie wymagań niefunkcjonalnych. Poniżej wypisano listę podstawowych wymagań tego typu.
% TO DO: wymagania powinny być w jakiś sposób walidowalne (muszą istnieć jakieś kryteria ich spełnienia). Jak więc można ocenić, czy poniższe wymagania rzeczywiście zostaną spełnione?
% TO DO: zwykle w wymaganiach niefunkcjonalnych mówi się o technologiach użytych, o przewidywanym wolumenie danych, o przewidywanych obciążeniach, o wymaganej infrastrukturze itp.

\subsection{Ogólnodostępność}

\begin{itemize}
	\item \textbf{Kryterium} Ogólnodostępność -- system powinien być dostępny dla każdego użytkownika mającego dostęp do internetu. Powinien być zgodny z przeglądarkami internetowymi.
	\item \textbf{Mierzalne kryteria}
		\begin{enumerate}
			\item czas odpowiedzi dla żądania użytkownika nie przekracza 2 sekund,
			\item kompatybilność z przeglądarkami: Google Chrome, Mozilla Firefox, Microsoft Edge,
			\item responsywność interfejsu użytkownika na różnych urządzeniach (desktop, smartphone).
		\end{enumerate}
	Wykorzystanie biblioteki React \ref{tab:zestawienie_narzędzi} umożliwi spełnieni kryterium ogólnodostępności.
		
\subsection{Niezawodność}
	\item \textbf{Niezawodność} -- system powinien być niepodatny na awarie.
	\item \textbf{Mierzalne kryteria}
		\begin{enumerate}
			\item średni czas pomiędzy zaobserwowanymi awariami powinien wynosić co najmniej 30 dni,
			\item średni czas naprawy w przypadku awarii nie powinien przekraczać 1 godziny.
		\end{enumerate}
	Niezawodność jest związana z działaniem aplikacji na środowisku chmurowym. 
	
	\subsection{Bezpieczeństwo}
	
	\item \textbf{Bezpieczeństwo} -- System powinien zabezpieczać dane użytkowników i posiadać mechanizmy uwierzytelniania i autoryzacji.
	\item \textbf{Mierzalne kryteria}
		\begin{enumerate}
			\item wrażliwe dane użytkowników przechowywane są w formie zaszyfrowanej,
			\item mechanizmy sprawdzania bezpieczeństwa haseł użytkowników.
		\end{enumerate}
		
		Wykorzystanie logiki napisanej w aplikacji serwerowej w języku JAVA \ref{tab:zestawienie_narzędzi} powinno umożliwiać szyfrowanie i sprawdzenie bezpieczeństwa haseł
		
		\subsection{Wydajność}
		\item \textbf{Wydajność} -- System powinien zapewniać satysfakcjonujący czas oczekiwania na wykonanie operacji.
	\item \textbf{Mierzalne kryteria}
		\begin{enumerate}
			\item większość operacji użytkownika powinno być wykonane poniżej 3 sekund,
			\item skalowalność systemu: możliwość obsłużenia równocześnie zalogowanych 100 użytkowników.
		\end{enumerate}
		
		Java, JavaScript umożliwiają szybkie wykonywanie operacji po stronie serwerowej i klienckiej.
		
		\subsection{Koszty}
	\item \textbf{Koszty} -- System powinien mieć niskie koszty utrzymania oraz być łatwy w utrzymaniu.
	\item \textbf{Mierzalne kryteria}
		\begin{enumerate}
			\item miesięczne koszty utrzymania nie powinny przekraczać określonego budżetu,
			\item wprowadzenie zmian przez deweloperów w systemie nie powinno generować dużych kosztów.
		\end{enumerate}
		
	Koszty utrzymania są związane z wdrożeniem systemu na środowisko produkcyjne. Dokumentacja systemu umożliwia szybką identyfikacje potencjalnych problemów co powinno zredukować koszty.
\end{itemize}
 










\pdfbookmark[0]{Skróty}{skroty.1}% 
%%\phantomsection
%%\addcontentsline{toc}{chapter}{Skróty}
\chapter*{Skróty}
\label{sec:skroty}
\noindent\vspace{-\topsep-\partopsep-\parsep} % Jeśli zaczyna się od otoczenia description, to otoczenie to ląduje lekko niżej niż wylądowałby zwykły tekst, dlatego wstawiano przesunięcie w pionie
\begin{description}[labelwidth=*]
  \item [SSMS] (\.ang \emph{SQL Server Managment Studio})
	\item [JPA] (\.ang \emph{Java Persistance API})
	\item [FR] (\.ang \emph{Functional Requirements})
	\item [ORM] (\.ang \emph{Object Relational Mapping})
	\item [REST] (\.ang \emph{Representational State Transfer})
	\item [HTTP] (\.ang \emph{Hypertext Transfer Protocol})
	\item [CSS] (\.ang \emph{Cascading Style Sheets})
	\item [API] (\.ang \emph{Application Programming Interface})
	\item [SEO] (\.ang \emph{Search Engine Optimization})
\end{description}

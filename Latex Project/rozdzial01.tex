\chapter{Wstęp}
\section{Wprowadzenie}
Dzięki dynamicznemu rozwoju technologii informacyjnych oraz znaczącemu postępowi w projektowaniu i wytwarzaniu elektroniki na rynku pojawiają się coraz to nowsze usługi i produktu. Dzieje się to sposób niemal ciągły, zapewniając przy tym skok cywilizacyjny. Jednak taki skok, oprócz niewątpliwych korzyści, generuje również skutki uboczne. Do takich należy między innymi problem szybkiego starzenia się sprzętu komputerowego. Problem ten widać szczególnie dobrze w firmach z branży IT, które wyposażają swoich pracowników w komputery, laptopy itp. Dążąc do zwiększenia zysków oraz obniżenia kosztów własnych menadżerowie tych firm muszą zapewnić pracownikom odpowiedni warsztat pracy, co pociąga za sobą konieczność dokonywania okresowej wymiany sprzętu oraz jego utylizacji.

Wymiana i utylizacja sprzętu zwykle jest uwarunkowana jego zdatnością do użytkowania. I tu nie chodzi tylko fizyczną sprawność poszczególnych urządzeń. Często wymieniany i utylizowany sprzęt jest w pełni funkcjonalny, ale z uwagi na zmiany w obszarze technologii, nie da się go już wykorzystać na tzw.\ ,,produkcji''. Taki sprzęt szkoda poddawać automatycznej kasacji. Dużo lepszym pomysłem jest jego odsprzedaż. 

Pomysł odsprzedaży wysłużonego sprzętu można zrealizować na różne sposoby. Jednym z nich jest uruchamianie loterii adresowanej do pracowników danej firmy. Sposób ten daje on możność osiągnięcia kliku korzyści: uzyskania częściowego zwrotu kosztów poniesionych na zakup sprzętu, zwiększenia poziomu integracji członków firmy, wdrożenia mechanizmów motywacyjnych itp.

% TO DO: warto byłoby pokazać jakieś przykłady podobnych systemów (systemów aukcyjnych sprzętu wycofywanego z firmy) - by dało się coś zacytować.

Spisy stanu posiadania częstokroć prowadzone są w postaci edytowalnych tabel. Do ich obsługi wykorzystuje się narzędzia typu Office (a dokładniej -- arkuszy kalkulacyjnych). W podstawowym scenariuszu do przygotowania i przeprowadzenia loterii wystarczy sięgać do odpowiedniego arkusza kalkulacyjnego i nanieść w nim odpowiednie zmiany.
Bywa, że informacje dotyczące danej loterii jest rozproszona pomiędzy kilkoma arkuszami. Omówiony sposób nie jest jednak najlepszy. Dużo lepszym byłoby tu uruchomienie osobnej aplikacji, działającej zgodnie z przyjętymi w firmie regułami. Chęć zaimplementowania takiej aplikacji stała się motywem do zdefiniowana tematu oraz celu niniejszej pracy.

   
\section{Cel projektu}
Celem pracy jest zaprojektowanie i zaimplementowanie narzędzia, które umożliwi ocenę stanu technicznego urządzeń wycofywanych z użytku, pozwoli wycenić ich wartość oraz posłuży jako wsparcie podczas realizacji procesu odsprzedaży. 

Projektowane narzędzie powinno pozwalać na inwentaryzowanie sprzętu, tj.\ umożliwiać zapis stanu technicznego urządzeń oraz ich ewentualne usterki (co wymaga zwykle fizycznych oględzin). W trakcie realizacji pracy będzie trzeba dobrze określić model i zakres przechowywanych danych, biorąc pod uwagę fakt, iż większość inwentaryzowanych sprzętów będzie posiadać jakąś specyfikację komponentów składowych, a te powinny być skatalogowane wraz z cennikiem. Na cenę końcową sprzętu powinien wpływać również jego 
wiek i czas użytkowania, jak również usterki czy uszkodzenia. Można więc rozważyć scenariusz, w którym do wyprzedawanego sprzętu dołączane są egzemplarze, które uległy awarii (i ich naprawa, z punktu widzenia firmy, nie jest opłacalna).


Ponadto narzędzie powinno zapewniać dostęp do informacji wszystkim osobom zaangażowanym: administratorowi (odpowiedzialnemu za edycję rekordów w podległych bazach danych),
jak również zwykłym użytkownikom (pracownikom zatrudnionym w firmie, zainteresowanym zakupem wyprzedawanego sprzętu). 

Zwykli użytkownicy powinni otrzymać opcję zapisu na loterię (zakup używanego sprzętu z przekazaniem środków finansowych firmie) jak również zapisu na charytatywną licytację (zakup używanego sprzętu z przekazaniem środków finansowych na jakiś cel charytatywny). Szczegóły realizacji obu opcji powinny zostać opracowane na etapie analizy wymagań. W ogólnym zarysie sprzedaż powinna być uruchamiana dopiero wtedy, gdy pula zgromadzonego sprzętu będzie odpowiednio liczna. 
Rozpoczęcie sprzedaży powinno być anonsowane użytkownikom. Wygrywający loterię lub licytację powinni otrzymywać informację o zakupionym sprzęcie wraz z instrukcjami jego odbioru. Narzędzie powinno również brać pod uwagę dane historyczne -- tj.\ powinno zwiększać szanse na wylosowanie sprzętu tym użytkownikom, którzy wcześniej nic nie wylosowali. W przypadku kiedy jest więcej sprzętu niż pracowników, pracownicy mogą wylosować kilka sprzętów.

Realizacja pracy powinna odbywać się z wykorzystaniem wybranego stosu technologicznego platformy \textbf{Java}. Planowane jest posłużenie się frameworkami \textbf{Springboota} i \textbf{Hibernate}. Interfejs użytkownika stworzonego narzędzia powinien przyjąć postać aplikacji webowej wyświetlanej w oknie przeglądarki internetowej. Do budowy frontendu wykorzystany ma być język \textbf{javascript}. Warstwa danych narzędziem powinna być zaimplementowana z wykorzystaniem \textbf{MS SQL Server} (zarządzanie tą bazą danych odbywać się może z poziomu  \textbf{SSMS} (\textbf{SQL Server Managment Studio}). Do testowania żądań HTTP może być wykorzystany \textbf{Postman}. Końcowy produkt powinien dać się łatwo wdrożyć w środowisku chmurowym.



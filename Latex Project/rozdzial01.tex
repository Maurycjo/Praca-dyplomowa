\chapter{Wstęp}
\section{Wprowadzenie}

Pracując w firmach informatycznych gdzie pracownikom wydaje się sprzęt służbowy natknąłem się na problem zarządzania takim sprzętem. Firmy sprzedają lub utylizują taki sprzęt jeżeli nie jest on zgodny do użytku lub przekroczył czas dozwolonego użytkowania. Częstym rozwiązaniem pozbycia się niechcianego sprzętu jest jego odsprzedaż. Ciekawym rozwiązaniem było by wykorzystanie loterii żeby określić kto może kupić określony sprzęt. Przy odkupowaniu sprzętu przez pracowników korzystają dwie strony, firmy pozbywają się sprzętu przy tym odzyskując trochę pieniędzy, a pracownicy kupują sprzęt po zaniżonej cenie. 

Doświadczenie pokazuje że pracownicy bardzo chętnie odkupują taki sprzęt, mogą z niego użytkować lub wykorzystać działające części komputera do złożenia nowego. Firmy odsprzedające sprzęt często prowadzą spis i zapisy na loterię w programie Excel. Wiele osób musi się napracować by mogła się odbyć taka loteria. W swoim projekcie chciałbym jak najbardziej zoptymalizować takie przedsięwzięcie. Chciałbym zaznaczyć, że częstym rozwiązaniem niedziałającego sprzętu komputerowego jest jego naprawa przez firmę oraz że istnieją dedykowane systemy do zgłaszania najróżniejszych usterek przez pracowników czym nie będzie się zajmować ten system. Jednak w przypadku awarii która byłaby nieopłacalna do naprawy można byłoby wykorzystać mój system do sprzedaży takiego urządzenia.

Stworzony system przechowywałby dane dotyczące inwentaryzowanego sprzętu. Inwentaryzacja często odbywa się fizycznie, trzeba określić stan techniczny urządzenia oraz jego ewentualne usterki. Każda osoba w firmie ma dostęp do tego systemu z tym że tylko administrator może dodawać i edytować nowe rekordy do bazy danych. W trakcie określania stanu technicznego administrator może się zalogować na system na sprawdzanym komputerze gdzie ma opcję automatycznego dodania parametrów komputera do bazy danych. W przypadku braku możliwości włączenia komputera możliwe jest ręczne wpisanie tych danych. Administrator wpisuje też swoje uwagi dotyczące sprzętu oraz jego gotowość do odsprzedaży. Jako że większość komputerów w firmie posiada zbliżoną specyfikacje w bazie danych będą ceny odpowiednich podzespołów które będą pomagały oszacowaniu ceny sprzedawanego sprzętu. Cena sprzętu mogłaby by być również określana na podstawie czasu użytkowania. Administrator zawsze może dostosować cenę ręcznie na podstawie usterek fizycznych.

Użytkownicy logujący się na system mogliby zapisywać się na loterię klikając odpowiedni przycisk i potwierdzając chęć wzięcia udziału. Loterie odbywałby się wtedy kiedy odpowiednia liczba sprzętu zostałaby zgromadzone. Administrator wtedy by tworzył losowanie. Po losowaniu użytkownik by widział jaki sprzęt wylosował oraz instrukcje jego odbioru. System by pamiętał poprzednie losowanie aby zwiększyć szanse w następnym losowaniu pracownikom którym nie udało się nic wylosować. W przypadku kiedy jest więcej sprzętu niż pracowników, pracownicy mogą wylosować kilka sprzętów. Losowaniu mogły by być też stacje dokujące i tablety.

Drugim alternatywnym sposobem zakupu używanego sprzętu mogłyby być licytacje charytatywne. Sprzęty mogły by być licytowane w systemie przez użytkowników, a pieniądze ze sprzedaży trafiły by na cele charytatywne.
   
\section{Cel projektu}

Celem projektu jest zaprojektowanie i zaimplementowanie narzędzia które umożliwi określenie stanu technicznego i wyceny urządzenia w celu jego odsprzedaży. Projekt będzie umożliwiał wykonanie losowania urządzeń w celu odsprzedaży lub zrobienie akcji charytatywnej. W systemie będą dwie rolę, standardowego użytkownika oraz administratora który zarządza danymi. Projekt optymalizuje wykorzystanie czasu w którym administrator systemu podaje dane oraz w którym użytkownicy zapisują się na losowanie.

 Wykorzystanie bazy danych umożliwi lepszą kontrolę stanu technicznego urządzeń. Wykorzystanie \textbf{Springboota i Hibernate }umożliwi łatwą komunikację z bazą danych. Technologia \textbf{Java }wraz z tymi frameworkami bardzo dobrze nadają się do tego tematu. Wykorzystanie aplikacji webowej wpłynie na łatwość użycia tego rozwiązania przez użytkowników. Wystarczy że użytkownik będzie miał dostęp firmowy do aplikacji która może zostać wdrożona na chmurę. Jako że Java nie umożliwia wczytywanie danych komputera możliwe będzie wywołanie skryptów powershella przez frontend. Frontend zostanie napisany w javascript(być może react). Narzędziem do zarządzania lokalną bazą danych będzie \textbf{SSMS }czyli \textbf{SQL Server Managment Studio}. Do testowania requestów http zostanie wykorzystany \textbf{Postman}. Ostatecznie rozwiązanie przyjęłoby formę która da łatwo się wdrożyć na chmurowe środowisko produkcyjne firmy.
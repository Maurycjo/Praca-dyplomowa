\chapter{Rozdział 2}
\section{Wymagania funkcjonalne}
\begin{enumerate}
	\item W systemie powinny być dostępne 2 rodzaje (role) użytkowników: pracownik i administrator
	\item Pracownik jest osobą zatrudnioną w firmie i powinien mieć uprawnienia do przeglądania sprzętów które są dostępne w losowaniu
	\item Pracownik powinien móc zapisywać się do wzięcia udziału w loterii
	\item Pracownik jest użytkownikiem który może założyć konto z tym że konto może być założone tylko na email firmowy
	\item Pracownik wybiera sprzęt do którego chce wziąć udział w loterii
	\item Administrator powinien dodawać do bazy danych sprzęty które będą dostępne w losowaniu
	\item Administrator może edytować wszystkie parametry sprzętu z bazy danych przez aplikacje
	\item Administrator może usuwać urządzenia z bazy danych
	\item Administrator ma wgląd w historię puli losowań
	\item Administrator określa przez aplikacje pulę losowania, w tej puli odbywa się wiele losowań
	\item Administrator może archiwizować sprzedane urządzenia
	\item Administrator może dodawać komponenty komputera do bazy
	\item Administrator określa jakie sprzęty znajdują się w puli i kiedy ta pula się odbywa
	\item System umożliwia autouzupełnianie pól dotyczących dodawanego sprzętu przez administratora przy pomocy bazy danych
	\item System pomaga w szacowaniu ceny sprzętu na podstawie komponentów urządzeń które znajdują się w bazie danych, z tym że to administrator ustala cenę końcową
	\item System umożliwia export wybranych danych sprzętu przy pomocy pliku JSON z tym że opis stanu technicznego musi określić samodzielnie
	\item System ma opcjonalną opcję włączaną przez administratora która będzie zwiększała prawdopodobieństwo wylosowania sprzętu przez użytkowników którzy w poprzednim puli losowaniu nic nie kupili
	\item System ma opcjonalną opcję która zmniejsza prawdopodobieństwo na wylosowanie kolejnego sprzętu przez pracownika który w tej puli już coś wylosował
	\item System ma opcjonalną opcję która sprawia że jeżeli jest w puli losowania więcej pracowników niż sprzętów to pracownicy którzy już coś wylosowali nie będą brani pod uwagę
	\item Na każdy sprzęt jest jedna loteria w której biorą udział zapisani pracownicy
	\item System po losowania wysyła automatyczny mail do zwycięscy loterii
	\item Pracownik w aplikacji potwierdza lub odrzuca chęć zakupu urządzenia, w przypadku kiedy w wyznaczonym czasie nie potwierdzi chęci zakupu system traktuje to jako odrzucone
	\item Odrzucone sprzęty mogą być znowu losowane w nowej puli


\end{enumerate}


\section{Wymagania niefunkcjonalne}
\begin{enumerate}
	\item W przypadku odrzucenia chęci zakupu sprzętu przez pracownika lub przez system, sprzęt może być losowany ponownie
	\item Losowanie odbywa się na podstawie parametrów ustawionych przez administratora
	\item System nie umożliwia płatności internetowej
	\item Administrator ostatecznie określa przez aplikacje czy pracownik odebrał sprzęt
	\item Przeglądanie systemu odbywa się przez aplikacje webową
	\item Użytkownik z konta administratora nie może brać udziału w losowaniu
	\item Może odbywać się wiele puli losowań w tym samym czasie

\end{enumerate}

\chapter{Analiza wymagań}
\section{Przypadki użycia}


% template %
\newcommand\addrow[2]{#1 & #2\\ \hline}

\newcommand\additemizedrow[2]{#1 &
        \begin{tabenum}
            #2
        \end{tabenum}
        \\ \hline}

% making stuff convenient %
\newcommand\name[1]{\addrow{Nazwa}{#1}}
\newcommand\actor[1]{\addrow{Aktor}{#1}}
\newcommand\udescription[1]{\addrow{Opis}{#1}}
\newcommand\precondition[1]{\addrow{Warunki wstępne}{#1}}
\newcommand\scenario[1]{\additemizedrow{Kroki}{#1}}
\newcommand\result[1]{\addrow{Wynik}{#1}}
\newcommand\extend[1]{\additemizedrow{Extend}{#1}}
\newcommand\includee[1]{\additemizedrow{Include}{#1}}

\newenvironment{usecase}{\tabularx{\textwidth}{|0{wl{3cm}}|0{X}|}\hline}{\endtabularx}
\setlength{\parindent}{0em}
\setlength{\parskip}{1em}


\subsubsection{Rejestracja nowego użytkownika}

\begin{usecase}
		\udescription{Rejestracja użytkownika ze standardową rolą, przy użyciu maila, nazwy użytkownika, Imienia, nazwiska, hasła i potwierdzenia hasła}
    \actor{Użytkownik}
    \precondition{Użytkownik z zadanym emailem oraz loginem nie jest zarejestrowany w systemie}
    \scenario{
        \item Podanie w formularzu maila, nazwy użytkownika, Imienia, nazwiska, hasła i potwierdzenia hasła
        \item Kliknięcie przycisku zarejestruj
      
    }
    \result{Użytkownik został zarejestrowany w systemie}
  
        
    
\end{usecase}

\subsubsection{Logowanie użytkownika}
\begin{usecase}
		\udescription{Logowanie użytkownika przy pomocy emaila lub nazwy użytkownika oraz hasła}
    \actor{Pracownik, Administrator}
    \precondition{Użytkownik został wcześniej zarejestrowany w systemie}
    \scenario{
        \item Podanie w formularzu nazwy użytkownika lub email oraz podanie hasła
        \item Kliknięcie przycisku zaloguj
    }
    \result{System sprawdza poprawność danych logowania, jeżeli są poprawne to następuje logowanie do systemu i nadanie odpowiednich uprawnień}
   
\end{usecase}

\subsubsection{Dodanie urządzenia przez administratora}
\begin{usecase}
		\udescription{Administrator może dodawać 3 różne rodzaje urządzeń zdefiniowanych w systemie}
    \actor{Administrator}
    \precondition{Administrator został wcześniej zarejestrowany w systemie i posiada uprawnienia administratora}
		\extend{
			\item filtrowanie urządzeń po nazwie biura
			\item filtrowanie urządzenia po ich typie, lub wyświetlanie wszystkich
			\item dodanie nowego procesora
			\item dodanie nowej pamięci ram
			\item dodanie nowej pamięci dyskowej
		}
    \scenario{
        \item Kliknięcie przycisku odpowiedzialnego za dodanie urządzenia
        \item Wybranie z listy rozwijanej typu urządzenia które będzie dodawane
				\item System wyświetla odpowiedni rodzaj urządzenia
				\item Wpisanie oraz wybranie odpowiednich parametrów reprezentujących urządzenie
				\item Dodanie urządzenia do systemu
    }
    \result{Urządzenie z ustawionymi parametrami zostało dodane do systemu}
   
\end{usecase}

\subsubsection{Modyfikacja urządzenia przez administratora}
\begin{usecase}
		\udescription{Administrator może modyfikować urządzenia}
    \actor{Administrator}
    \precondition{Administrator został wcześniej zarejestrowany w systemie i posiada uprawnienia administratora}
		\extend{
			\item filtrowanie urządzeń po nazwie biura
			\item filtrowanie urządzenia po ich typie, lub wyświetlanie wszystkich
			\item dodanie nowego procesora
			\item dodanie nowej pamięci ram
			\item dodanie nowej pamięci dyskowej
		}
    \scenario{
        \item Kliknięcie przycisku modyfikuj na odpowiednim wierszu reprezentującym urządzenie
				\item System wykrywa odpowiedni rodzaj urządzenia i wyświetla formularz z poprzednimi danymi urządzenia
				\item Wpisanie oraz wybranie odpowiednich parametrów reprezentujących urządzenie
				\item Modyfikacja urządzenia
    }
    \result{Urządzenie z ustawionymi parametrami zostało zmodyfikowane}
   
\end{usecase}

\subsubsection{Usunięcie urządzenia przez administratora}
\begin{usecase}
		\udescription{Administrator może usuwać urządzenia}
    \actor{Administrator}
    \precondition{Administrator został wcześniej zarejestrowany w systemie i posiada uprawnienia administratora}
		\extend{
			\item filtrowanie urządzeń po nazwie biura
			\item filtrowanie urządzenia po ich typie, lub wyświetlanie wszystkich
		}
    \scenario{
        \item Kliknięcie przycisku usuń na odpowiednim wierszu reprezentującym urządzenie
    }
    \result{Urządzenie zostało usunięte z systemu}
   
\end{usecase}

\subsubsection{Ustawienie gotowości sprzętu do losowania}
\begin{usecase}
		\udescription{Administrator zarządza urządzeniem i określa jego gotowość do loterii}
    \actor{Administrator}
    \precondition{Administrator został wcześniej zarejestrowany w systemie i posiada uprawnienia administratora, Urządzenie nie zostało wcześniej wylosowane w loterii}
		\extend{
			\item filtrowanie urządzeń po nazwie biura
			\item filtrowanie urządzenia po ich typie, lub wyświetlanie wszystkich
		}
    \scenario{
        \item Kliknięcie przycisku gotowości do losowania na odpowiednim wierszu reprezentującym urządzenie
    }
    \result{Urządzenie jest gotowe do losowania, pracownicy mogą zapisywać się na loterię danego urządzenia}
   
\end{usecase}

\subsubsection{Ustawienie statusu sprzętu na sprzedane lub niesprzedane}
\begin{usecase}
		\udescription{Administrator kontroluje które sprzęty zostały już sprzedane i dostarczone do pracownika}
    \actor{Administrator}
    \precondition{Administrator został wcześniej zarejestrowany w systemie i posiada uprawnienia administratora, Urządzenie zostało wylosowane w loterii}
		\extend{
			\item filtrowanie urządzeń po nazwie biura
			\item filtrowanie urządzenia po ich typie, lub wyświetlanie wszystkich
		}
    \scenario{
        \item Kliknięcie przycisku odpowiedzialnego za zmianę statusu sprzedane na odpowiednim wierszu reprezentującym urządzenie
    }
    \result{Zmiana statusu urządzenia na sprzedane lub niesprzedane}
   
\end{usecase}

\subsubsection{Wyświetlanie listy uczestników biorącym udział w losowaniu}
\begin{usecase}
		\udescription{Administrator sprawdza jacy użytkownicy biorą udział w losowaniu sprzętu}
    \actor{Administrator}
    \precondition{Administrator został wcześniej zarejestrowany w systemie i posiada uprawnienia administratora, Status urządzenia został ustawiony na gotowe do losowania}
		\extend{
			\item filtrowanie urządzeń po nazwie biura
			\item filtrowanie urządzenia po ich typie, lub wyświetlanie wszystkich
		}
    \scenario{
        \item Kliknięcie przycisku odpowiedzialnego za wyświetlanie listy uczestników na odpowiednim wierszu reprezentującym urządzenie
    }
    \result{Wyświetlanie widoku uczestników losowania, oraz ich statusu dotyczącego czy już wygrali loterię}
   
\end{usecase}

% TODO więcej przypadków użycia oraz odwołania pomiędzy tymi przypadkami
% skorzystanie z include żeby je uprościć
